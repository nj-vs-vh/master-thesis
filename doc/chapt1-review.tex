\chapter{Методы регистрации и восстановления параметров ШАЛ}

\label{chapt:review}

В этой главе приводится краткий обзор современных методов регистрации ШАЛ на примере крупнейших установок: Pierre Auger Observatory, Telescope Array, Якутской установки по регистрации ШАЛ; обозреваются методы восстановления энергии первичной частицы, применяемые в этих экспериментах, и ассоциированные с ними неопределённости. Обсуждается важность оценки систематических погрешностей для корректной интерпретации результатов эксперимента, кратко описываются источники этой погрешности.


\section{Методы регистрации ШАЛ}

По регистрируемой компоненте ШАЛ детекторы делятся на детекторы заряженных частиц и детекторы сопутствующих оптических излучений ливня (методы регистрации радиоизлучений не включены в этот обзор). Крупные современные эксперименты используют одновременно несколько каналов регистрации. Так, Pierre Auger Observatory использует наземные детекторы (баки, заполненные водой и снабжённые несколькими ФЭУ для регистрации черенковского света, рождаемого в воде электронами и мюонами) и несколько флуоресцентных телескопов, наблюдающих за площадью эксперимента со стороны \cite{Abraham2010}. Telescope Array использует сходную методику наземных детекторов (в качестве них используются сцинтилляторы, а не водяные баки, как в PAO) и флуоресцентных телескопов \cite{AbuZayyad2013}. В Якутской установке ШАЛ используются наземные сцинтилляторы и подземные мюонные счётчики, а также детекторы черенковского излучения, расположенные массивом на поверхности земли \cite{Ivanov2007}.


\section{Методы и оценка погрешности восстановления энергии}

Энергия первичной частицы является ключевым параметром ливня, а энергетический спектр, получаемый на основе оценки энергий множества событий, является важнейшим результатом для любого эксперимента по наблюдению ШАЛ. С физической точки зрения форма спектра несёт информацию о процессах рождения, ускорения и распространения космических лучей в Галактике и Метагалактике.

Для восстановления энергии разработано множество теоретических методов, в этом разделе приведён их краткий обзор вместе с соответствующими экспериментами. Отдельное внимание уделяется оценке систематических погрешностей, возникающих в каждом из методов.

Систематические погрешности восстановления энергии являются важным фактором, ограничивающим наши знания о спектре космических лучей в области сверхвысоких энергий. Спектры разных экспериментов дают значения, отличающиеся друг от друга на десятки процентов, что может быть феноменологически объяснено систематическими сдвигами в шкалах энергий \cite{wg2013}. Поэтому в современной экспериментальной технике особое место уделяется процедурам калибровки детекторов и перекрёстной проверки разных оценок.

Приведём для примера методику восстановления энергии в эксперименте Pierre Auger Observatory.

Низкоуровневая калибровка наземных детекторов PAO проводится к <<вертикальным эквивалентным мюонам>> (Vertical Equivalent Muons, VEM) \cite{Bertou2006}. Эта величина показывает, как подсказывает название, число гипотетических мюонов, движущихся по вертикали и одновременно попавших в объём детектора, которые породили бы такой же сигнал в системе регистрации. Калибровка к этим величинам происходит онлайн силами электроники каждой наземной станции. В качестве источника калибровки используются атмосферные мюоны, дающие характерный пик в распределении интенсивностей импульсов.

Далее каждому для события рассчитывается значение $S_{38}$ -- это флюенс заряженных частиц на расстоянии $1000~\text{м}$ от оси ливня в случае, если бы ливень имел зенитный угол $38^{\circ}$ \cite{PAO-ICRC-2005}. Таким образом определённая величина учитывает тот факт, что для ливней, пришедших под значительными углами, детекторы массива будут расположены на разных глубинах. Эмпирически измеренная величина $S_{38}$ оказывается скоррелирована с энергией, восстановленной по данным флуоресцентных телескопов, поэтому из неё можно провести оценку для каждого конкретного экспериментального события (включая и те, для которых нет соответствующих измерений флуоресцентного света).

Неопределённость такой процедуры весьма велика и происходит из нескольких факторов. Так, погрешность восстановления $S_{38}$ составляла порядка $10 \%$ \cite{Collaboration2005}, погрешность перевода этой величины в энергию -- ещё около $20 \%$.
