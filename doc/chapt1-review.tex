\chapter{Методы регистрации и восстановления параметров ШАЛ}

\label{chapt:review}

В этой главе приводится краткий обзор современных методов регистрации ШАЛ на примере крупнейших установок: Pierre Auger Observatory, Telescope Array, Якутской установки по регистрации ШАЛ; описывается пример метода восстановления энергии первичной частицы и ассоциированные с ними неопределённости. Кроме того, обсуждается важность оценки систематических погрешностей для корректной интерпретации результатов эксперимента.


\section{Методы регистрации ШАЛ}

По регистрируемой компоненте ШАЛ детекторы делятся на детекторы заряженных частиц и детекторы сопутствующих оптических излучений ливня (методы регистрации радиоизлучений не включены в этот обзор). Крупные современные эксперименты используют одновременно несколько каналов регистрации. Так, Pierre Auger Observatory использует наземные детекторы (баки, заполненные водой и снабжённые несколькими ФЭУ для регистрации черенковского света, рождаемого заряженными частицами в воде) и несколько флуоресцентных телескопов, наблюдающих за площадью эксперимента со стороны \cite{Abraham2010}. Telescope Array использует сходную методику наземных детекторов (в качестве них используются сцинтилляторы, а не водяные баки, как в PAO) и флуоресцентных телескопов \cite{AbuZayyad2013}. В Якутской установке ШАЛ используются наземные сцинтилляторы и отдельные счётчики мюонов, расположенные под землёй для исключения электронной компоненты, а также детекторы черенковского излучения, расположенные массивом на поверхности земли \cite{Ivanov2007}.


\section{Методы и оценка погрешности восстановления энергии}

Энергия первичной частицы является ключевым параметром ливня, а энергетический спектр, получаемый на основе оценки энергий множества событий, является важнейшим результатом для любого эксперимента по наблюдению ШАЛ. Для восстановления энергии разработано множество теоретических методов, но общая идея их заключается в том, что, зная физические закономерности, управляющие развитием ливня, можно предсказать долю энергии, переданную той или иной компоненте или излучению ливня. Измеряя эту энергию экспериментально, можно с некоторой погрешностью перейти назад к суммарной энергии всего ливня, которая и равна энергии первичной частицы.

Отдельное внимание следует уделить оценке погрешностей, возникающих при восстановлении энергии. Систематические погрешности являются важным фактором, ограничивающим наши знания о спектре космических лучей в области сверхвысоких энергий. Так, спектры, полученные в разных экспериментах, дают значения, отличающиеся друг от друга иногда на десятки процентов, что представляет серьёзную проблему, если речь действительно идёт о разнице потоков. В то же время это расхождение может быть феноменологически объяснено систематическими сдвигами в шкалах энергий разных экспериментов, после устранения которых спектры совпадают с высокой точностью, не только по абсолютным значениям, но и по форме особенностей \cite{wg2013}. Поэтому в современной экспериментальной технике особое место уделяется процедурам калибровки детекторов и перекрёстной проверки разных оценок.

Приведём далее пример такого метода -- восстановление энергии по данным наземных детекторов эксперименте Pierre Auger Observatory. Этот пример не стремится быть исчерпывающим или полностью актуальным, но дать иллюстрацию и общее представление о методике.

Низкоуровневая калибровка наземных детекторов PAO проводится к величине <<вертикальных эквивалентных мюонов>> (Vertical Equivalent Muons, VEM) \cite{Bertou2006}. Эта величина показывает, как подсказывает название, число гипотетических мюонов, движущихся по вертикали и одновременно попавших в объём детектора, которые породили бы такой же сигнал в системе регистрации. Калибровка для перевода измеренного сигнала в единицы VEM происходит онлайн силами электроники каждой наземной станции. В качестве калибровочного источника используются атмосферные мюоны, дающие характерный пик в распределении интенсивностей импульсов. Эта примечательно простая процедура сразу позволяет оперировать сигналами, выраженными в физических единицах.

Далее для каждого экспериментального события рассчитывается значение $S_{38}$ -- это флюенс (плотность потока, интегрированна по времени прохождения ливня) заряженных частиц на расстоянии $1000~\text{м}$ от оси ливня в случае, если бы ливень имел зенитный угол $38^{\circ}$ \cite{PAO-ICRC-2005}. Такое  специфичное определение, приводящее потоки к конкретному зенитному углу, учитывает тот факт, что для наземных ливней детекторы массива будут расположены на разных глубинах. Далее по данным гибридных событий, то есть зарегистрированных одновременно наземными детекторами и флуоресцентными телескопами, проводится кросс-калибровка: $S_{38}$ оказывается скоррелирована с энергией, восстановленной по данным флуоресцентных телескопов $E_{FD}$. Построив зависимость по гибридным событиям, можно получить оценку энергии по наземным детекторам и для экспериментальных событий без соответствующих измерений флуоресцентного света.

Неопределённость такой процедуры весьма велика и происходит из нескольких факторов. С одной стороны, погрешность реконструкции $S_{38}$ составляла порядка $10 \%$ \cite{Collaboration2005}, погрешность перевода этой величины в энергию -- ещё около $20 \%$ \cite{Roth2007}.
