\chapter{Оценка параметров ШАЛ}

В предыдущей главе описан метод байесовской деконволюции и его применение к данным эксперимента СФЕРА-2. В результате получена безмодельная оценка потоков фотонов на мозаике ФЭУ в том смысле, что она не зависит от детальных предположений о свойствах и источниках света, падающего на мозаику, кроме самых общих представлений. В этой главе на основе полученного результата, а также качественного представления о пространственно-временной структуре сигнала ШАЛ реконструируется функция пространственного распределения (ФПР) черенковского света. Наконец, на основе полученной ФПР делаются оценки параметров ШАЛ.

Стоит отметить, что методы, изложенные в этой главе, не являются инновационными сами по себе, напротив, использованы уже хорошо изученные подходы. Основной целью является демонстрация того, как эти методы работают в контексте эксперимента СФЕРА-2 с учётом описанной процедуры деконволюции и неопределённостей, которые она порождает.

\section{Выделение сигнала ШАЛ и оценка значимости}


\section{Оценка направления прихода ливня}


\section{Оценка положения оси ливня}


\section{Восстановление ФПР}

\subsection{Проекция полей зрения ФЭУ на плоскость ливня}

\subsection{Расчёт плотности черенковских фотонов}

До сих пор мы, как указано в разделе \ref{sec:photon-to-phels-conversion}, работали с количеством фотоэлектронов, рождённых сигналом ШАЛ на фотокатоде ФЭУ. Для построения ФПР черенковского излучения требуется, во-первых, перейти от фотокатода к поверхности, с соответствующим увеличением числа фотоэлектронов в $\approx 10^5$ раз (\textit{уточнить}), во-вторых, перейти от числа фотоэлектронов к числу, собственно, черенковских фотонов.

Для первой задачи ...

Для решения второй задачи заметим, что искомая величина плотности черенковского света измеряется \textit{на единицу энергии}: $Q(x, y)$ [$\text{фотоны} \cdot \text{м}^{-2} \cdot \text{эВ}^{-1}$] \cite{Budnev2005}, так как в соответствии с формулой Франка-Тамма \cite{Tamm1939} $dN/dE \approx Const$ в видимой части спектра. Тогда зная квантовую эффективность ФЭУ $\kappa(E)$, а точнее её среднее значение $\bar{\kappa}$ в диапазоне энергий $[E_{min}, E_{min} + \Delta E]$, можно перейти от флюенса эквивалентных фотоэлектронов $n_{ph. el.} (x, y)$ [$\text{фотоэлектроны} \cdot \text{м}^{-2}$] к величине $Q(x, y)$:

\begin{equation}
	Q(x, y) = \frac{n_{ph. el.}}{\Delta E \; \bar{\kappa}}
\end{equation}

