\chapter{Оценка параметров ШАЛ}

В результате проведения процедуры байесовской деконволюции получена безмодельная оценка потоков фотонов (в эквивалентных фотоэлектронах) на мозаике ФЭУ. Безмодельной она является в том смысле, что не зависит от детальных предположений о свойствах и источниках света, падающего на мозаику, кроме самых общих представлений. В этой главе на основе полученного результата, а также качественного представления о пространственно-временной структуре сигнала ШАЛ реконструируется функция пространственного распределения (ФПР) черенковского света. Наконец, на основе полученной ФПР делаются оценки параметров ШАЛ.

Стоит отметить, что методы, изложенные в этой главе, не являются новаторскими сами по себе, напротив, использованы уже хорошо изученные подходы. Основной целью является демонстрация того, как эти методы работают в контексте эксперимента СФЕРА-2 с учётом описанной процедуры деконволюции и неопределённостей, которые она порождает.

\section{Выделение сигнала ШАЛ и оценка значимости}

В процессе деконволюции не делается предположений о наличии или отсутствии сигнала ШАЛ в исследуемой области экспериментального кадра. Из моделирования ливня и оптической системы эксперимента известен общий вывод: фотоны ШАЛ достигают мозаики в виде \textit{пакетов} --- групп фотонов с приближённо нормальным распределением времён прихода. Каждый пакет можно описать тремя параметрами: $n_{EAS}$ --- суммарное число фотонов в пакете, $\mu_t$ --- среднее время прихода фотонов, $\sigma_t$ --- стандартное отклонение времён прихода. Ещё одним параметром является среднее число фоновых фотонов $\lambda_{n}$, однако, как показано в разделе \ref{sec:expdata-preparation-for-deconvolution}, оно известно из абсолютной калибровки сигнала.

\subsection{Выделение пакета фотонов ШАЛ}

\label{sec:signal-reconstruction}

Как и задача деконволюции (\ref{sec:bayesian-deconvolution-solution}), задача выделения сигнала ШАЛ может быть решена с помощью формализма байесовского вывода алгоритмом MCMC-сэмплирования. В качестве параметров модели используем $\Theta \equiv (n_{EAS}, \mu_t, \sigma_t)$, в качестве наблюдаемых данных --- полученную в результате деконволюции выборку значений $\vec{n}$.

Определим функцию правдоподобия для этой задачи: она должна давать вероятность того, что при фиксированном значении $\Theta$ будет получено наблюдаемое $\vec{n}$. То обстоятельство, что $\vec{n}$ измерен не прямо, а задан апостериорным распределением, легко учесть, просто усреднив значения функции правдоподобия по всем элементам этой выборки.

Представим $\vec{n}$ в виде суммы $\vec{n}_{EAS} + \vec{n}_{noise}$. Распределение случайного вектора $\vec{n}_{EAS}$ проще всего разыграть численно, генерируя выборки времён прихода фотонов объёмом $n_{EAS}$ из распределения $N(\mu_t, \sigma_t)$, и рассчитывая из них гистограмму в границах экспериментальных бинов. Для нахождения искомой функции правдоподобия остаётся вычислить вероятность того, что <<остаток>> фотоэлектронов $n_{noise}^{(i)}$ в каждом бине имеет пуассоновское распределение с $\lambda_{n}$:

\begin{equation}
	\mathcal{L}(\Theta) \equiv P(\vec{n} | n_{EAS}, \mu_t, \sigma_t) = \prod_{i} \frac{e^{-\lambda_n} \lambda_n^{n_{noise}^{(i)}}}{(n_{noise}^{(i)})!}
\end{equation}

Формула выше описывает значения правдоподобия при фиксированных $\vec{n}_{EAS}$ и  $\vec{n}$, поэтому для получения окончательного результата требуется усреднить значение $\mathcal{L}(\Theta)$ по соответствующим распределениям (флуктуациям гистограммы $\vec{n}_{EAS}$ и апостериорному распределению деконволюции).

В отличие от неинформативного априорного распределения, описанного для деконволюции в разделе \ref{sec:deconv-prior}, для $\Theta$ можно выбрать осмысленные априорные распределения из данных моделирования. Известно, что $n_{EAS}$ для интересующего нас диапазона энергий в $1$ - $100$ ПэВ имеет априорное распределение, экспоненциально спадающее от нуля с показателем $\sim 40$, а $\sigma_t$ --- $N(2.4, 1)$, обрезанное в нуле. Для $\mu_t$ было выбрано априорное распределение, равномерное на ширине окна деконволюции \footnote{В будущем при развитии методики можно уже на этапе поиска пакета <<угадывать>> его предполагаемое положение из приближённой оценки ориентации плоскости ливня по нескольким самым ярким каналам, и вносить эту информацию в априорное распределение $\mu_t$.}. Использование информативных априорных распределений помогает в процессе сэмплирования быстрее <<навестись>> на нужные области, например, сразу отбросить слишком широкие или многочисленные пакеты как маловероятные.

Техническая реализация MCMC-сэмплирования полностью аналогична описанной в разделе \ref{sec:mcmc-sampling}.

\begin{figure}
	\centering
	\includegraphics[width=\columnwidth]{signal-reconstruction-example}
	\caption{Выделение пакета фотонов ШАЛ из результатов байесовской деконволюции. Фиолетовыми кривыми представлены $10$ пакетов фотонов, соответствующих случайным элементам из апостериорной выборки $\Theta$ (см. текст). Видно, что пакеты группируются вблизи ожидаемого пика, но их параметры варьируются, учитывая неопределённость данных деконволюции.}
	\label{pic:signal-reconstruction-example}
\end{figure}


\subsection{Оценка значимости выделенного сигнала}

Одно из преимуществ полностью статистического подхода --- возможность использовать понятие значимости в процессе разделения сигнала ШАЛ и сигнала от фоновых фотонов. Широко принятый в байесовской статистике инструмент для этого --- байесовский информационный критерий (Bayesian information criterion, BIC), введённый Шварцем \cite{Schwarz1978} и представляющий собой в некотором смысле информационный критерий Акаике \cite{Akaike1974}, адаптированный для байесовского анализа. Суть его состоит в следующем: при наличии нескольких моделей, описывающих данные, их можно сравнить по количеству информации, которое теряется при замене данные на модель. Чем меньше потеря информации, тем меньше будет значение критерия, и тем лучше показывает себя модель. Вычисление проводится по формуле

\begin{equation}
	\mathrm{BIC} = k \ln n - 2 \ln \mathcal{L}_{max}
\end{equation}

Здесь $k$ --- число параметров модели, $n$ --- число элементов выборки данных, $\mathcal{L}_{max}$ --- максимальное значение функции правдоподобия для данной модели. Структура выражения указывает на важное качество критерия: чем больше число параметров модели $k$, тем больше требуемый прирост $\mathcal{L}_{max}$, это позволяет предотвратить переобучение модели с большим числом параметров.

Для определения значимости найденного сигнала ШАЛ нам нужно сравнить две модели: модель <<только шума>> ($n_{EAS} = 0$) вовсе без параметров ($k=0$), и модель <<шум + сигнал>>, описанную в предыдущем разделе, которая имеет $k=3$ параметра. По разнице $ \Delta \mathrm{BIC} = \mathrm{BIC}_{noise} - \mathrm{BIC}_{noise + EAS}$ можно судить о значимости восстановленного сигнала.

Если $\Delta \mathrm{BIC} \leq 0$, то модель только шума оказывается более состоятельной, и такой канал можно удалить из анализа. Если $0 < \Delta \mathrm{BIC} < 4$, то сигнал можно считать слабо значимым \cite{Kass1995}, на практике оказывается, что в эту область чаще всего попадают артефакты деконволюции или сильные флуктуации фона. Каналы, в которых $\Delta \mathrm{BIC} > 4$, принимались как достоверные (хотя и в этом случае иногда находятся артефакты, которые позже отсеиваются на этапе восстановления геометрии ливня).

На рис. \ref{pic:deconvolution-and-reconstruction} представлен пример сначала деконволюции, а затем восстановления параметров пакетов фотонов для экспериментального события. Оценка значимости сигналов использована, чтобы часть точек отсеить совсем, а часть --- пометить как <<сомнительные>>, однако видно, что в последних каналах присутствуют, по-видимому, артефакты деконволюции, которые дают сигналы высокой значимости, не ложащиеся в картину ливня. Их, впрочем, легко отделить по времени прихода, этот процесс будет описан в следующей главе. Тем не менее, наличие подобных ложных срабатываний затрудняет поиск истинных слабых сигналов ШАЛ, способы борьбы с ними разрабатываются --- в первую за счёт повышения вычислительной устойчивости процедуры деконволюции --- и будут применены в будущем.

\begin{figure}
	\centering
	\includegraphics[width=0.86\columnwidth]{deconvolution-and-reconstruction}
	\caption{Деконволюция и восстановление параметров пакетов для экспериментального события \#10675. Для параметров $n_{EAS}, \mu_t, \sigma_t$ ярким цветом показаны точки со значениями $\Delta \mathrm{BIC} > 4$, тусклым --- $\Delta \mathrm{BIC} \in (0, 4]$, значения с отрицательным $\Delta \mathrm{BIC}$ исключены, видно, что они соответствуют <<пустым>> областям кадра.}
	\label{pic:deconvolution-and-reconstruction}
\end{figure}


\section{Направление прихода ливня}

Стандартный метод оценки направления прихода (или, иначе, ориентации оси) ливня основан на представлении о том, что фронт черенковского света (а в случае с другими установками --- и заряженных частиц) с хорошей точностью является плоским. Поэтому, аппроксимируя плоскостью экспериментально измеренную зависимость времени прихода фронта $\bar{t}_{gnd}(x, y)$, мы сразу получаем оценку углов ориентации вектора нормали.

Для обработки данных эксперимента СФЕРА-2 требуется сделать дополнительный шаг: учесть время распространения света от снега до детектора. Это легко сделать, учитывая, что для каждого ФЭУ известен центр его поля зрения на поверхности $(x_i, y_i)$ (см. раздел \ref{sec:light-collection-from-surface} и в частности рис. \ref{pic:experimental-pmt-fov-example}), и отсюда, зная высоту подъёма установки $H$, получаем $\bar{t}_{i} = \mu_t^{(i)} - \frac{\sqrt{H^2 + x_i^2 + y_i^2}}{c}$.

Задача аппроксимации трёхмерных точек плоскостью решается линейным методом наименьших квадратов, однако сам по себе этот метод неустойчив к выбросам, а, как видно на рис. \ref{pic:deconvolution-and-reconstruction}, выбросы в данных $\mu_t$ присутствуют. Для решения этой проблемы была разработана методика итеративного фитирования: аппроксимируется набор точек, находится самая удалённая от плоскости точка, выбрасывается, новый набор аппроксимируется заново, и так далее. Процедура повторяется до тех пор, пока угол между векторами нормали в двух последовательных аппроксимациях не станет меньше заданного наперёд значения. Значение допустимого <<дрожания>> было положено равным $0.1^{\circ}$. Следует заметить, что это не ограничение на погрешность определения ориентации оси, но только на устойчивость оптимального значения этой ориентации относительно удаления <<наихудшей>> точки. Погрешность определения углов $\theta$ и $\phi$ получается естественным образом в процессе фитирования, зависит от числа точек и составляет в общем случае порядка нескольких градусов. На рис. \ref{pic:plane-reconstruction} показаны два примера этой процедуры для разных событий.


\begin{figure}
	\centering
	\includegraphics[width=0.45\columnwidth]{showe-plane-approximation-10675}
	\hfill
	\includegraphics[width=0.45\columnwidth]{showe-plane-approximation-10685}
	\caption{Восстановление ориентации ШАЛ для двух экспериментальных событий. Красными показаны точки, автоматически исключённые в процессе итеративного фитирования (см. текст).}
	\label{pic:plane-reconstruction}
\end{figure}

\subsection{Перспективы уточнения оценки}

Описанный метод является стандартным и устоявшимся, однако может быть уточнён с учётом индивидуальных особенностей эксперимента СФЕРА-2. В бакалаврской дипломной работе \cite{bachelorsthesis} был разработан метод уточнения оценки ориентации оси, рассматривающий ширину пакета фотонов как вторичный показатель. Идея метода основана на том, что для протяжённых полей зрения ФЭУ ожидается наличие зависимости $\sigma_t(x, y)$ --- для ФЭУ, через которые фронт ливня проходит первым, пакет фотонов будет \'{у}же, чем для расположенных на противоположной стороне мозаики. Этот <<геометрический эффект>> является следствием взаимной угла между падающим и рассеянным черенковским светом, и соответствующим сжатием или растяжением пакета во времени.

В настоящую работу такой анализ не вошёл, поскольку процедура требует отдельной модификации для работы с новыми данными, однако будет внесён в общий алгоритм обработки данных в дальнейшем.

\section{Положение оси ливня}

Для оценки положения оси ливня, то есть координат её пересечения с поверхностью наблюдения, можно использовать ряд методов; в этом разделе описан один из простейших, основанный только на предположении о том, что функция пространственного распределения черенковских фотонов монотонно убывает с расстоянием от оси ливня.

Из восстановленных параметров пакетов фотонов в каждом канале мы имеем зависимость оценки $n_{EAS}$, найденной для каждого канала в разделе \ref{sec:signal-reconstruction}, от координат на поверхности наблюдения. К этому моменту мы исключили из рассмотрения сигналы с низкой значимостью, а также сигналы, не вошедшие по времени прихода в восстановленную плоскость ливня (см. предыдущий раздел).

Сформулируем функцию правдоподобия, задающую вероятность того, что ось пересекает поверхность наблюдения в точке $(x_{ax}, y_{ax})$. Зная эти координаты, перенумеруем каналы по возрастанию расстояния от предполагаемого положения оси --- тогда функция правдоподобия будет равна вероятности того, что $\forall i, j: i < j$ будет выполняться $n_{EAS}^{(i)} > n_{EAS}^{(j)}$. Для простоты положим здесь и далее, что апостериорное распределение $n_{EAS}^{(i)}$ в $i$-том бине приближённо нормально, то есть полностью описывается средним $\mu_i$ и стандартным отклонением $\sigma_i$.

\begin{equation}
	\mathcal{L}(x_{ax}, y_{ax}) = \prod_i \prod_{j > i} \left( 1 - F_{N}(0, \mu_i - \mu_j, \, \sqrt{\sigma_i + \sigma_j}) \right)
\end{equation}

Здесь $F_{N}(x, \mu, \sigma) = \frac{1}{2} \left[ 1 + \mathrm{erf} \left( \frac{x - \mu}{\sqrt{2 \sigma^2}} \right) \right]$ --- функция распределения гауссовой случайной величины, её параметры $\mu_i - \mu_j$ и $\sqrt{\sigma_i + \sigma_j}$ соответствуют распределению разности двух нормалных случайных величин, $1 - F_N(0)$ даёт вероятность того, что разность случайных величин больше нуля.

Эту функцию правдоподобия можно максимизировать для поиска наиболее вероятного положения оси. Следует заметить, что, так как в основе $\mathcal{L}$ лежит только представление о монотонности ФПР, но не о характере зависимости потока от радиуса, область максимального правдоподобия будет иметь конечные размеры, в её пределах упорядочивание каналов по возрастанию расстояния от оси не будет изменяться, и функция правдоподобия будет иметь строго одинаковые значения. На практике оказалось, что область такого <<вырождения>> весьма мала, и метод пригоден по крайней мере для начальной оценки.

Более сложные методы, вероятно, могут использовать описанный метод как первое приближение, но в дальнейшем восстанавливать положение оси вместе с формой ФПР в едином фитировании. Переход к таким методам не представляет концептуальной сложности.

\section{Восстановление функции пространственного распределения черенковского света}

После определения ориентации и положения оси ливня в пространстве можно приступить к оценке пространственного распределения черенковского света ШАЛ. Форма ФПР черенковского света представляет существенный интерес, так как даёт способ оценки параметров ливня, слабо зависящий от модели ядерного взаимодействия. В частности, нормировка ФПР даёт информацию об энергии ливня, а форма (часто измеряемая показателем наклона, определяемого как отношение потоков на двух радиусах), предоставляет возможность для определения массы первичной частицы ШАЛ \cite{Patterson1983, Dawson1989, TOKUNO2008}.

Полный обзор и сравнение разных аппроксимаций черенковской ФПР остаётся за рамками настоящей работы, для демонстрации используем двухпараметрическую функцию, разработанную для аппроксимации данных детекторов черенковского света в эксперименте Тунка-25 \cite{Budnev2005}:

\begin{equation}
	\label{eq:tunka-25-ldf}
	\begin{gathered}
	Q(R) = Q_{kn} \cdot \begin{cases}
		\exp \left( \frac{(R_{kn} - R) \cdot (1 + 3/(R+3))}{R_0} \right) \text{ при } R < R_{kn} \\
		\left(\frac{R_{kn}}{R}\right)^b \text{ при } R \geq R_{kn} 
	\end{cases} \\
	R_0 = 10^{2.95 - 0.245 P}, \text{м}  \\
	R_{kn} = 155 - 13P, \text{м} \\
	b = 1.19 + 0/23 P
	\end{gathered}
\end{equation}

Параметр $Q_{kn}$ задаёт нормировку ФПР, а $P$ называется параметром наклона, он равен отношению $Q(100)/Q(200)$ и определяет форму пространственного распределения.

Также стоит отметить, что возможен другой подход, основанный не на аналитической аппроксимации ФПР, а на прямом сравнении экспериментальных данных с ФПР, полученной из Монте-Карло симуляции ливня. Предполагается, что такой метод должен давать наибольшую точность, хотя и требовать больше вычислительных ресурсов. Вне зависимости от способа получения функции $Q(x, y)$ и количества её варьируемых параметров, способ сравнения с экспериментальными данными остаётся тем же.

\subsection{Проекция полей зрения ФЭУ на плоскость ливня}

В разделе \ref{sec:light-collection-from-surface} описан процесс получения <<полей зрения>> ФЭУ --- распределений коэффициентов сбора $f^{(i)}(x_{gnd}, y_{gnd})$ по отражающей поверхности под установкой. Однако функция $Q(x_{shw}, y_{shw})$ задаётся в плоскости ливня, поэтому требуется спроецировать $f^{(i)}$ на неё же.

Для этого учтём, что плоскость ливня задаётся нормалью $(\theta_{shw}, \varphi_{shw})$ в системе земли и для определённости пересекает поверхность земли в начале координат. На плоскости ливня можно ввести систему координат с горизонтальной осью $x_{shw}$, направленной <<слева-направо>> с точки зрения движущегося ливня, и осью $y_{shw}$, лежащей в одной плоскости с вертикалью (<<снизу-вверх>> с точки зрения ливня). Тогда для произвольной точки на поверхности земли, выраженной полярными координатами $(r_{gnd}, \varphi_{gnd})$ из простых геометрических соображений найдём $x_{shw} = - r_{gnd} \, \sin (\varphi_{shw} - \varphi_{gnd})$, $y_{shw} = - r_{gnd} \, \cos (\varphi_{shw} - \varphi_{gnd}) \, \cos \theta_{shw}$.

Таким образом спроецированные поля зрения дают суммарное распределение чувствительности установки $f(x_{shw}, y_{shw})$ в плоскости фронта ливня. Пример такой проекции для ливня с зенитным углом падения $\approx 30^{\circ}$ приведён на рис. \ref{pic:projected-pmt-fov}. Заметим, что, так как функция $f$ является безразмерным коэффициентом сбора, нет необходимости учитывать преобразование элемента площади при проекции.

\begin{figure}
	\centering
	\includegraphics[width=\columnwidth]{projected-pmt-fov}
	\caption{Распределение чувствительности установки в плоскости фронта ливня для события \% 10675.}
	\label{pic:projected-pmt-fov}
\end{figure}

\subsection{Преобразование фотонов черенковского света ШАЛ в эквивалентные фотоэлектроны}

До сих пор мы, как указано в разделе \ref{sec:photon-to-phels-conversion}, работали с количеством эквивалентных фотоэлектронов, рождённых сигналом ШАЛ на фотокатоде ФЭУ. Для построения ФПР черенковского излучения требуется описать, как происходит пересчёт одной величины в другую.

Заметим, что искомая величина плотности черенковского света измеряется \textit{на единицу энергии}: $Q(x, y)$ [$\text{фотоны} \cdot \text{м}^{-2} \cdot \text{эВ}^{-1}$] \cite{Budnev2005}, так как в соответствии с формулой Франка-Тамма \cite{Tamm1939} $dN _{\gamma}/dE \approx Const$ в видимой части спектра. Тогда зная квантовую эффективность ФЭУ $\kappa(E)$, а точнее её среднее значение $\bar{\kappa}$ в диапазоне энергий $[E_{min}, E_{min} + \Delta E]$, можно перейти от $Q$ к флюенсу эквивалентных фотоэлектронов $n_{ph. el.} (x, y)$ [$\text{фотоэлектроны} \cdot \text{м}^{-2}$]:

\begin{equation}
	\Delta E \; \bar{\kappa} Q(x, y) = n_{ph. el.}
\end{equation}

Кривые квантовой эффективности для двух видов ФЭУ приведены на рис. \ref{pic:pmt-quanteffs}. Коэффициент $\Delta E \; \bar{\kappa}$ составляет $0.34~\text{эВ}$ для Hamamatsu R3886 и $0.39~\text{эВ}$ для ФЭУ84-3.

\begin{figure}[H]
	\centering
	\includegraphics[width=\columnwidth]{pmt-quantum-efficiences}
	\caption{Кривые квантовой эффективности для двух видов ФЭУ, использованных в эксперименте СФЕРА-2}
	\label{pic:pmt-quanteffs}
\end{figure}


\subsection{Неопределённость процесса сбора света с поверхности}

Учтём, что до сих пор мы получали апостериорную оценку на число фотоэлектронов, выбитых с фотокатода ФЭУ под действием фотонов ШАЛ. В то же время из модели ливня, учитывая коэффициенты сбора света с поверхности снега, мы можем получить только математической ожидание этой величины, но не её точное значение --- процесс сбора света сам является стохастическим. Хорошим приближением для него является пуассоновское распределение. Иначе говоря, если среднее число фотонов, ожидаемое к $i$-том канале, равно $\lambda^{(i)}$, то реально зарегистрировано будет $n^{(i)} \sim \mathrm{Poisson}(\lambda^{(i)})$. Таким образом нам необходимо, зная апостериорное (относительно процесса деконволюции, выделения пакета фотонов и т.д.) распределение $n$, нужно получить оценку $\lambda$ (индекс канала $i$ далее для краткости опущен).

Сначала рассмотрим случай, когда $n$ известно точно. Тогда теорема Байеса сразу даёт апостериорное распределение $\lambda$:

\begin{equation}
	P(\lambda | n) = P(n | \lambda) = \frac{e^{-\lambda} \lambda^n}{n!}
	\label{eq:lmbda-posterior-precise}
\end{equation}

Иначе говоря, плотность апостериорного распределения непрерывной величины $\lambda$ задаётся той же формулой, что и само распределение пуассона, но где $\lambda$ является аргументом, а $n$ --- параметром.

Теперь перейдём к случаю, где $n$ не известно точно, но имеет некоторое распределение. Учитывая, что $n$ может принимать только целые значения, представим выборку из апостериорного распределения в виде таблицы частотности: $\lbrace n_i, p_i \rbrace, i = 1, ..., k$, где $p_i$ --- вероятность, что $n$ имеет значение $n_i$. Тогда ясно, что апостериорное распределение $\lambda$ будет равно дискретной <<свёртке>> выражения (\ref{eq:lmbda-posterior-precise}) с распределением $n$:

\begin{equation}
P(\lambda) = \sum_{i=1}^{k} p_i \frac{e^{-\lambda} \lambda^{n_i}}{n_i !}
\end{equation}

На практике оказывается, что такая свёртка уширяет апостериорное распределение $n$ не более чем на $10$ -- $20 \%$, сдвиг математического ожидания из-за асимметрии распределения Пуассона пренебрежимо мал (менее $1 \%$).


\subsection{Аппроксимация ФПР}

Пользуясь результатами педыдущих двух разделов, запишем окончательное выражение для ожидаемого среднего числа фотоэлектронов от света ШАЛ, зарегистрированных в $i$-том ФЭУ.

\begin{equation}
	\int_{\infty} f^{(i)}(x, y)
	; \Delta E \; \bar{\kappa} \; Q(x, y) \; dx dy = \lambda_n^{(i)}
\end{equation}

Интегрирование можно провести численно по известным точкам распределения $f^{(i)}(x, y)$ в плоскости ливня. Используя данные апостериорного распределения $\lambda_n^{(i)}$, можно найти среднее значение и стандартное отклонение этой величины для использования в аппроксимации.

Визуализировать такое фитирование в традиционных координатах $R, Q(R)$ непросто, так как величина $Q(R)$ сворачивается с двумерной функцией распределения чувствительности для каждого ФЭУ. Приближённую качественную картину можно получить, если пренебречь протяжённостью полей зрения и считать, что они достаточно малы, чтобы можно было положить $Q(x, y) = Const = Q(x_c, y_c)$: $\Delta E \; \bar{\kappa} \; Q(x_c, y_c) \int_{\infty} f^{(i)}(x, y) \; dx dy = \lambda_n^{(i)}$. Это приближение позволяет приписать каждому измерительному каналу значение приближённое экспериментальное значение $Q$, которые можно затем нанести на график и  визуально сравнить с теоретической кривой.

Такая визуализация приведена на рис. \ref{pic:ldf-fit-example-1} и \ref{pic:ldf-fit-example-2}, однако следует подчеркнуть её неточность. В частности за счёт протяжённых полей зрения, эксперимент СФЕРА-2 чувствителен к черенковскому свету в приосевой области, поэтому точность и устойчивость фита заметно выше, чем может показаться по этим упрощённым графикам. Также точность повышается за счёт наложения полей зрения ФЭУ: в некоторых областях поверхности свет регистрируется сразу двумя или даже тремя ФЭУ, что эффективно уменьшает погрешности определения $\lambda_n$ в них.


\begin{figure}
	\centering
	\includegraphics[width=\columnwidth]{ldf-fit-10685}
	\caption{Упрощённая визуализация фитирования ФПР по данным эксперимента СФЕРА-2 (см. текст). Погрешность определения нормировки $Q_{kn}$ показана на графике коридором вокруг наиболее вероятной кривой. Инструментальная погрешность определения энергии вычислена по формуле (\ref{eq:E0-from-Q175}) с учётом погрешностей $Q_{kn}$ и $P$.}
	\label{pic:ldf-fit-example-1}
\end{figure}

\begin{figure}
	\centering
	\includegraphics[width=\columnwidth]{ldf-fit-10675}
	\caption{То же самое, что рис. \ref{pic:ldf-fit-example-1}, для другого экспериментального события.}
	\label{pic:ldf-fit-example-2}
\end{figure}


\subsection{Определение параметров ливня}

Используя описанную процедуру фитирования для оценки оптимального значения и неопределённости каждого из параметров ФПР, можно, наконец, перейти к оценке параметров первичной частицы ливня. За рамками данной работы остаются вопросы связи энергии и массы первичной частицы, а также глубины максимума, с наблюдаемыми характеристиками ФПР. На этом этапе необходимо учитывать модельные неопределённости, то есть статистический характер зависимости параметров ФПР от исследуемых характеристик ливня.

В этом разделе ограничимся простым применением формулы для связи $E_0$ и $Q(R)$ \cite{Budnev2005}:

\begin{equation}
	\label{eq:E0-from-Q175}
	E \; [\text{ТэВ}] = 400 \cdot Q(175)^{0.95}
\end{equation}

Чтобы определить неопределённость $Q_{175}$, и как следствие $E$ с учётом оцененных неопределённостей параметров $Q_{kn}$ и $P$ сэмплируем каждый из них и вычислим среднее и стандартное отклонение по декартову произведению выборок (как видно из выражения (\ref{eq:tunka-25-ldf}), ФПР нетривиальным образом зависит от $P$).

Полученные энергии с соответствующими погрешностями приведены на рис. \ref{pic:ldf-fit-example-1} и \ref{pic:ldf-fit-example-2} вместе с параметрами ФПР. Погрешность определяется для каждого события отдельно, но в общем по двум примерам с различными энергиями и положениями оси относительно детектора её можно оценить в $5$ -- $10 \%$. Детальное исследование систематической инструментальной погрешности (в том числе её зависимость от энергии и положения ливня, высоты и ориентации установки) будет предметом дальнейших изысканий с использованием набранной статистики событий.

Также следует качественно сравнить полученные величины с модельными погрешностями, которые могут быть найдены, например, в работе \cite[табл. 1]{Anokhina2007}. Для ливней с энергией $1$ -- $10$ ПэВ флуктуации $Q(150)$, например, оцениваются в $2$ -- $10 \%$, уменьшаясь с энергией и массой первичной частицы. Для событий с энергиями $\approx 80$ и $\approx 206$ ПэВ эти модельные неопределённости будут ещё меньше --- следовательно, особенно важным становится именно точное определение инструментальной погрешности.

