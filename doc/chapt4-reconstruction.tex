\chapter{Оценка параметров ШАЛ}

В предыдущей главе описан метод байесовской деконволюции и его применение к данным эксперимента СФЕРА-2. В результате получена безмодельная оценка потоков фотонов на мозаике ФЭУ -- безмодельная в том смысле, что она не зависит от предположений о свойствах и источниках света, падающего на мозаику. В этой главе на основе полученного результата, а также качественного представления о пространственно-временной структуре сигнала ШАЛ реконструируется функция пространственного распределения (ФПР) черенковского света. Наконец, на основе полученной ФПР делаются оценки параметров ШАЛ.