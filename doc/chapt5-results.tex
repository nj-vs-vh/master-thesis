\chapter{Заключение}

В настоящей работе развиваются методы обработки данных эксперимента СФЕРА-2, нацеленные на восстановление детальной структуры ШАЛ. Основное направление развития состоит в построении более глубокой и детальной модели детектора и оценки инструментальных неопределённостей, сопряжённых с процессом регистрации черенковского света ШАЛ. Предложен новый, более модульный по сравнению с предшествующими работами, принцип построения процедуры анализа, призванный обеспечить её большую прозрачность и дать инструменты для перекрёстной проверки всех промежуточных выводов.

Для достижения поставленных целей реализован оригинальный метод деконволюции на основе байесовской статистики и метода Монте-Карло с марковскими цепями, приведены обоснования его корректности и примеры применения. Байесовский формализм для описания восстановленных величин с помощью их апостериорных распределений оказался удобным и во всём последующем анализе, и его применение предлагается расширять.

Результаты разработанных алгоритмов деконволюции и выделения сигнала ШАЛ следует считать предварительными: ещё предстоит провести тщательную проверку, оптимизацию, детальное сравнение с результатами предыдущих методов. Алгоритмы оценки параметров ливня и первичной частицы, приведённые в работе, также являются предварительными и служат в большей степени для проверки концепции.

Однако уже эти новые результаты представляются качественно верными. В частности, оценка относительной инструментальной погрешности определения энергии первичной частицы ШАЛ в $5$ -- $10 \%$ согласуется с ожидаемым значением, в частности, с систематическими погрешностями других экспериментов по регистрации ШАЛ. В то же время, так как это значение получается пособытийно, несколько примеров, приведённых в работе, несомненно, не могут дать полной картины.

В духе принципов открытой науки весь анализ проведён с помощью свободных программных инструментов с открытым исходным кодом: на языке \verb|Python| с использованием библиотек NumPy, SciPy, emcee. Все оригинальные коды, с помощью которых получены приведённые результаты, доступны в публичном git-репозитории \footfullcite{githubRepo}.
