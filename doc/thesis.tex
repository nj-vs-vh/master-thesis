\documentclass[12pt]{report}

\usepackage{lscape}		% Для включения альбомных страниц

\usepackage{amsmath,amsthm,amssymb}  %подгрузка пакетов для корректного и наиболее полного. должен грузиться как можно раньше, т.к. внутри себя определяет шрифты и некоторые прочие настройки, которые несколько не соответствуют тому, что надо.

\usepackage{mathtext} %позволяет использоваться текстовые кириллические индексы в формулах
\everymath{\displaystyle}   %отключаем мастабирование символов в многоэтажных дробях, индексах и пределах интегралов в соответствии с традицией русской типографской вёрстки

\usepackage{afterpage}  %данный покет вводит команду \afterpage, которая вставляет пустую страницу сразу за страницей, где эта команда встретилась. Удобна для аккуратного расположения иллюстраций в дипломах и диссертациях.

\usepackage[utf8]{inputenc}
\usepackage[T2A,T1]{fontenc}
% \usepackage{lmodern} % несколько иное начертание латиницы

\usepackage[main=english, russian]{babel}
\usepackage{csquotes}

\usepackage[left=30mm, right=20mm, top=20mm, bottom=20mm]{geometry}   %поля по требованиям факультета (с поправкой, что там ограничения на поля снизу), пакет geometry воспринимает размеры в mm, cm, in и пр.

% \usepackage{indentfirst}   %заставляет ТеХ делать отступ в первом абзаце главы/параграфа/раздела

\usepackage{setspace}   %пакет позволяет выставлять полуторные интервалы в тексте, но не вмешивается в подписи, таблицы и сноски

\usepackage{sectsty}
\renewcommand{\cleardoublepage}{}
\renewcommand{\clearpage}{\vfill}

\usepackage[
	backend=biber,
	sorting=none,  % в порядке цитирования в тексте
]{biblatex}
\addbibresource{thesis.bib}

\usepackage{float}
\usepackage{graphicx}
\usepackage[inline]{enumitem}
\usepackage{booktabs}
\usepackage{multirow,makecell,array}	% Улучшенное форматирование таблиц

\renewcommand{\rmdefault}{ftm} % включаем Times New Roman
%искусственное переопределение размера шрифтра с 12 на 14.
\renewcommand{\tiny}{\fontsize{7}{8.4pt}\selectfont}
\renewcommand{\scriptsize}{\fontsize{9}{11pt}\selectfont}
\renewcommand{\footnotesize}{\fontsize{11}{13.6pt}\selectfont}
\renewcommand{\small}{\fontsize{12}{14.5pt}\selectfont}
\renewcommand{\normalsize}{\fontsize{14}{18pt}\selectfont}
\renewcommand{\large}{\fontsize{17}{20pt}\selectfont}
\renewcommand{\Large}{\fontsize{20}{25pt}\selectfont}
\onehalfspacing   %полуторный интервал

\renewcommand{\labelenumi}{\arabic{enumi}.} 
\renewcommand{\labelenumii}{\arabic{enumi}.\arabic{enumii}}
\renewcommand{\labelenumiii}{\arabic{enumi}.\arabic{enumii}.\arabic{enumiii}}
\renewcommand{\labelenumiv}{\arabic{enumi}.\arabic{enumii}.\arabic{enumiii}.\arabic{enumiv}}

\usepackage{hyperref}    %позволяет сделать в полученной PDFке рабочими ссылки как в оглавлении, так и в списке литературы и вообще в тексте

% ./pic is used for dynamically updated images -- python scripts
% write their output there
% when the image is finalized it is manually copied to ./pic/final
% and scripts can do whatever they want
\graphicspath{{./pic/final/} {./pic/external/} {./pic/}}

\setcounter{secnumdepth}{5}
\newcommand{\cov}{\mathrm{cov}}

\usepackage{titlesec}[compact]

\usepackage{subcaption}
\usepackage{graphicx}

% \captionsetup[figure]{name=Figure}

% \titlelabel{\thetitle.\quad}

\titleformat{\chapter}[hang]{\bfseries\Large}{\thechapter}{1ex}{}
\titleformat{\section}[hang]{\bfseries\large}{\thesection}{1ex}{}
\titleformat{\subsection}[hang]{\bfseries}{\thesubsection}{1ex}{}

\titleformat{\paragraph}[runin]
{\hyphenpenalty=10000 \itshape}
{\theparagraph.}
{1ex}{}[.]

\titleformat{\subparagraph}[runin]
{}
{\thesubparagraph.}
{1ex}{}[.]

% \setlength{\parindent}{1.25cm} %настройка отступов
\sloppy   %убираем висячие строки  и подобное безобразие
\clubpenalty=10000		% Запрещаем разрыв страницы после первой строки абзаца
\widowpenalty=10000		% Запрещаем разрыв страницы перед последней строкой абзаца


\usepackage{microtype}
\SetProtrusion
{
	encoding = T2A,
	family = faq
}
{
	« = {1000,   },
	» = {  , 1000},
	„ = {1000,   },
	“ = {  , 1000},
	( = {1000,   },
	) = {  , 1000},
	! = {  , 1000},
	? = {  , 1000},
	: = {  , 1000},
	; = {  , 1000},
	. = {  , 1000},
	- = {  , 500},
	{,}= {  , 1000}
}


% \title{Байесовская деконволюция для эксперимента СФЕРА-2}

\author{Igor Vaiman}

\date{\today}

\begin{document}
	
	\thispagestyle{empty}

\begin{center}
%ФЕДЕРАЛЬНОЕ ГОСУДАРСТВЕННОЕ БЮДЖЕТНОЕ ОБРАЗОВАТЕЛЬНОЕ\\
%УЧРЕЖДЕНИЕ ВЫСШЕГО ОБРАЗОВАНИЯ\\
%<<МОСКОВСКИЙ ГОСУДАРСТВЕННЫЙ УНИВЕРСИТЕТ \\
%имени М.В. ЛОМОНОСОВА>>\\

%\vspace{5mm}

%ФИЗИЧЕСКИЙ ФАКУЛЬТЕТ 

%\vspace{5mm}

%КАФЕДРА ФИЗИКИ КОСМОСА
%\end{center}

%\vfill
%\begin{center}
%{МАГИСТЕРСКАЯ ДИССЕРТАЦИЯ}
%\end{center}


\begin{center}
\Huge
Reconstruction of the axis position and primary particle energy of extensive air showers in the SPHERE-2 experiment \\
\vspace{3mm}
\large
Extended abstract
\end{center}

\vspace{5mm}

Vaiman Igor, SINP MSU

%\begin{flushright}
%Выполнил студент

%214м группы

%Вайман Игорь Алексеевич

%\vspace{15mm}

%Научный руководитель:

%доцент Подгрудков Дмитрий Аркадьевич
%\end{flushright}

%\vspace{10mm}

%\begin{flushleft}
%Допущена к защите 22.04.2021

%Зав.кафедрой  

%\end{flushleft}
%\vfill
%\begin{center}
%Москва

%2021
\end{center}

%\newpage  %титульный лист
	
	% \tableofcontents  %оглавление
	
	\chapter*{Introduction}

Cosmic rays (CR) are highly energetic charged particles that traverse space. They play a crucial role in understanding the dynamics of space systems. Their energy spectrum spans $12$ orders of magnitude ($10^9$ to $10^{21}$ eV), providing insights into their origin and propagation within the Solar System, Galaxy, and Metagalaxy. Above $10^{15}$ eV, indirect methods are employed to study cosmic rays through the detection of extensive air showers (EAS), utilizing Earth's atmosphere as a giant calorimeter.

The SPHERE-2 experiment focuses on detecting EAS using the relatively novel method of reflected Cherenkov light. This work enhances the experiment's data processing by incorporating a deeper understanding of the SPHERE-2 detector gained in recent years. The key improvement lies in modeling the detector as a stochastic system, enabling accurate estimation of instrumental uncertainties. Bayesian statistics serves as the primary mathematical framework for the analysis.

	% \chapter{Методы регистрации и восстановления параметров ШАЛ}

\label{chapt:review}

В этой главе приводится краткий обзор современных методов регистрации ШАЛ на примере крупнейших установок: Pierre Auger Observatory, Telescope Array, Якутской установки по регистрации ШАЛ; приводится пример метода восстановления энергии первичной частицы и ассоциированные с ними неопределённости. Кроме того, обсуждается важность оценки систематических погрешностей для корректной интерпретации результатов эксперимента.


\section{Методы регистрации ШАЛ}

По регистрируемой компоненте ШАЛ установки можно разделить на детекторы заряженных частиц и детекторы сопутствующих оптических излучений ливня (методы регистрации радиоизлучений ШАЛ, также активно развиваемые, оставлены за рамками данной работы). Крупные современные эксперименты используют одновременно несколько каналов регистрации. Так, Pierre Auger Observatory использует наземные детекторы (баки, заполненные водой и снабжённые несколькими ФЭУ для регистрации черенковского света, рождаемого заряженными частицами в воде) и несколько флуоресцентных телескопов, наблюдающих за площадью эксперимента со стороны \cite{Abraham2010}. Telescope Array использует сходную методику наземных детекторов (в качестве них используются сцинтилляторы, а не водяные баки, как в PAO) и флуоресцентных телескопов \cite{AbuZayyad2013}. В Якутской установке ШАЛ используются наземные сцинтилляторы и отдельные счётчики мюонов, расположенные под землёй для исключения электронной компоненты, а также детекторы черенковского излучения, расположенные массивом на поверхности земли \cite{Ivanov2007}. Во всех этих экспериментах активно используются перекрёстные проверки для сравнения и взаимной калибровки разных каналов регистрации.


\section{Оценка энергии первичной частицы и её неопределённость}

Энергия первичной частицы является ключевым параметром ливня, а энергетический спектр, получаемый на основе оценки энергий множества событий, является важнейшим результатом для любого эксперимента по наблюдению ШАЛ. Для восстановления энергии разработано множество теоретических методов, но общая идея их заключается в том, что, зная физические закономерности, управляющие развитием ливня, можно предсказать долю энергии, переданную той или иной компоненте либо излучению ливня. Измеряя эту долю энергии экспериментально, можно с некоторой погрешностью перейти назад к полной энергии всего ливня, которая, в свою очередь, даст оценку энергии первичной частицы.

Отдельное внимание следует уделить оценке погрешностей, возникающих при восстановлении энергии. Систематические погрешности являются важным фактором, ограничивающим наши знания о спектре космических лучей в области сверхвысоких энергий. Так, спектры, полученные в разных экспериментах, дают значения, отличающиеся друг от друга иногда на десятки процентов. Этот факт представляет серьёзную проблему, если интерпретировать его как действительную разницу потоков. В то же время это расхождение может быть феноменологически объяснено систематическими сдвигами в шкалах энергий разных экспериментов, после устранения которых спектры совпадают с высокой точностью, причём не только по абсолютным значениям, но и по форме особенностей спектра \cite{wg2013}. Поэтому в современной экспериментальной технике особое место уделяется процедурам калибровки детекторов и оценке неопределённостей.

Приведём далее пример такого метода --- восстановление энергии по данным наземных детекторов эксперименте Pierre Auger Observatory. Этот пример не стремится быть исчерпывающим или полностью актуальным, но дать иллюстрацию и общее представление о методике.

Низкоуровневая калибровка наземных детекторов PAO проводится к величине <<вертикальных эквивалентных мюонов>> (Vertical Equivalent Muons, VEM) \cite{Bertou2006}. Эта величина показывает, как подсказывает название, число гипотетических мюонов, движущихся по вертикали и одновременно попавших в объём детектора, которые породили бы такой же сигнал в системе регистрации. Калибровка для перевода измеренного сигнала в единицы VEM происходит онлайн силами электроники каждой наземной станции. В качестве калибровочного источника используются атмосферные мюоны, дающие характерный пик в распределении интенсивностей импульсов. Эта примечательно простая процедура сразу позволяет оперировать сигналами, выраженными в физических единицах.

Далее для каждого экспериментального события рассчитывается значение $S_{38}$ --- это флюенс (плотность потока, интегрированна по времени прохождения ливня) заряженных частиц на расстоянии $1000~\text{м}$ от оси ливня в случае, если бы ливень имел зенитный угол $38^{\circ}$ \cite{PAO-ICRC-2005}. Такое  специфичное определение, приводящее потоки к конкретному зенитному углу, учитывает тот факт, что для наземных ливней детекторы массива будут расположены на разных глубинах. Далее по данным гибридных событий, то есть зарегистрированных одновременно наземными детекторами и флуоресцентными телескопами, проводится кросс-калибровка: $S_{38}$ оказывается скоррелирована с энергией, восстановленной по данным флуоресцентных телескопов $E_{FD}$. Построив зависимость двух величин по гибридным событиям, можно получить оценку энергии по наземным детекторам и для экспериментальных событий без соответствующих измерений флуоресцентного света.

Неопределённость такой процедуры весьма велика и происходит из нескольких факторов. Так, погрешность реконструкции $S_{38}$ составляла порядка $10 \%$ \cite{Collaboration2005}, а погрешность перевода этой величины в энергию первичной частицы ливня --- ещё около $20 \%$ \cite{Roth2007}.

	\chapter{Эксперимент СФЕРА-2}

В этой главе приводится идея, мотивация и краткая история развития метода регистрации отражённого черенковского света, который развивается в эксперименте СФЕРА-2; обозреваются проведённые измерения и полученные на сегодняшний день результаты, обосновывается необходимость и основные направление дальнейшей работы; наконец, приводится качественная модель процесса регистрации ШАЛ на этом эксперименте.

\section{Метод регистрации отражённого черенковского света ШАЛ}

\textbf{TBD}: идея Чудакова, Наварро, Тянь-Шань, см. \cite{chernov2015-overview} (эта же работа на англ. \cite{Sphere2015}). Написать про мотивацию для метода: статьи с теоретическими критериями для определения массового состава ШАЛ, например \cite{Anokhina2007}, \cite{Chernov2017-ICRC}

\section{Предварительные результаты}

\textbf{TBD}: см. \cite{Antonov2013}, 

\section{Качественная модель работы детектора}

Опишем на качественном уровне теоретические и методические представления, которые лежат в основе интерпретации данных СФЕРА-2. Для этого проследим, что происходит от развития каскада в атмосфере и до получения файлов с экспериментальными данными.

\subsection{Развитие ШАЛ в атмосфере}

Сколько-нибудь полный обзор теории ШАЛ, и даже черенковского света ШАЛ в отдельности, лежит за рамками данной работы, поэтому изложим только самые общие представления. Широкий атмосферный ливень -- каскад вторичных частиц, вызванный взаимодействием первичной частицы большой энергии с атмосферой -- развивается в виде тонкого диска заряженных частиц -- адронов, мезонов, лептонов, гамма-квантов. Черенковский свет -- одно из сопутствующих излучений ШАЛ, возникающее при движении заряженных частиц со скоростью, превышающую скорость света в среде распространения. Черенковский свет ШАЛ также распространяется в виде тонкого -- до нескольких метров в толщину -- диска, ориентированного перпендикулярно оси ливня. Для практических задач моделирование процессов развития ШАЛ проводится численно, например, с помощью программы CORSIKA \cite{CORSIKA-report}.

Уже на этапе моделирования ШАЛ закладывается ряд неопределённостей реконструкции параметров первичной частицы. Во-первых, модели ядерного взаимодействия при энергиях $\gtrapprox 10^{16}~\text{эВ}$ не проверены экспериментально, являясь, в сущности, экстраполяциями, и разные такие модели могут приводить к несколько разным картинам развития ливня для одной и той же первичной частицы; во-вторых, развитие ШАЛ -- принципиально стохастический процесс, что приводит к статистической неопределённости любой реконструкции. Эти неопределённости будем в дальнейшем называть \textit{модельными}, так как они напрямую связаны с моделью описываемого явления, в противоположность \textit{инструментальным} неопределённостям, связанным с процессом измерения характеристик ливня.

\subsection{Отражение света от снега}
\label{sect:snow-reflection}

Эксперименты, использующие отражённый черенковский свет, предъявляют особые требования к земной поверхности на уровне наблюдения -- она должна служить <<экраном>>, который бы равномерно и предсказуемо рассеивал свет. В случае СФЕРЫ-2 таким экраном служит заснеженный лёд озера Байкал.

Рассеивающие свойства снега могут существенно повлиять на работу эксперимента, поэтому для измерения и мониторинга этих условий были приложены определённые усилия. В частности, были проведены прямые измерения коэффициента отражения в зависимости от угла \cite[рис.~11]{Sphere2015}, и было обнаружено, что зависимость хорошо согласуется с законом рассеяния Ламберта для идеальной диффузной поверхности: \textit{яркость} рассеянного света не зависит от угла, то есть интенсивность имеет чисто геометрическую зависимость $I \propto \cos \theta_n$ \cite{Antonov2019}. Альбедо $a$ -- отношение падающего и отражённого потоков -- было принято независимым от длины волны \cite{Warren1982} и равным $0.9$.

\subsection{Сбор отражённого света}

Установка СФЕРА-2 поднималась аэростатом на высоту $400$ -- $900~\text{м}$, для сбора света с поверхности использовалась оптическая система из диафрагмы, сферического зеркала и мозаики ФЭУ, которая схематично изображена на рис. \ref{pic:sphere-detector-optical-scheme}. В результате каждый из ФЭУ обозревал область на поверхности диаметром $10$ -- $50~\text{м}$.

\begin{figure}
	\centering
	\includegraphics[width=\columnwidth]{optical_scheme}
	\caption{Оптическая схема детектора СФЕРА-2}
	\label{pic:sphere-detector-optical-scheme}
\end{figure}

Качественно оценим коэффициент сбора света: пусть на участок поверхности в окрестности точки $(x, y)$ в системе координат с центром в проекции детектора падает $\delta N_{gr}$ фотонов, и пусть горизонтально ориентированная диафрагма радиусом $R_{d}$, поднята на высоту $H$. Обозначая угол рассеяния света от нормали $\theta_n$, можно оценить число фотонов, которое достигнет диафрагмы из результатов предыдущего раздела и геометрических соображений

\begin{equation}
	\delta N_{d} = \frac{R_d^2 \cos \theta_n}{H^2 + x^2 + y^2} K \cos \theta_n \delta N_{gr}
\end{equation}

Из этого простого расчёта при характерных значениях $H = 600~\text{м}$, $R \approx 0.45~\text{м}$ получается оценка общего коэффициента сбора установки: $(0.5 \div 1) \cdot 10^{-6}$.

Для расчёта количества фотонов, достигающего каждого отдельного ФЭУ, требуется численное моделирование распространения света внутри мозаики, отражения от зеркала, поглощения тыльной стороной мозаики и элементами конструкции. Результатом такого моделирования является коэффициент сбора как функция координат на поверхности, или \textit{чувствительность} каждого ФЭУ $k_i(x, y)$. Более строго, если флюенс черенковских фотонов равен $n(x, y)$, то число ожидаемое число фотонов, попавших в $i$-тый ФЭУ, будет равно $\int_{\infty} k_i(x, y) n(x, y) dx dy$.

Функции $k_i$ имеют форму пятен, соответствующих полям зрения ФЭУ на поверхности льда. Пример нескольких таких пятен, а также суммарной чувствительности всей мозаики для одного из экспериментальных событий изображён на рис. \ref{pic:experimental-pmt-fov-example}. Ясно, что поля зрения зависят от высоты подъёма и ориентации установки, поэтому при расчёте учитывается данные GPS и инклинометра соответственно.

\begin{figure}
	\centering
	\includegraphics[width=\columnwidth]{experimental-pmt-fov-example}
	\caption{Пример моделирования сбора света с поверхности установкой СФЕРА-2 для экспериментального события \#10699. Показаны чувствительности трёх отдельных ФЭУ и суммарная чувствительность всей мозаики. Учтены данные о высоте подъёма и наклоне установки -- поэтому картина чувствительности сдвинута относительно проекции установки и слегка вытянута.}
	\label{pic:experimental-pmt-fov-example}
\end{figure}

\subsection{Регистрация собранного света}

Финальный этап процесса регистрации события включает несколько подэтапов, которые мы изложим особенно подробно, поскольку именно они обуславливают многие исследуемые в настоящей работе эффекты.

\subsubsection{Мозаика ФЭУ}
\label{sec:pmt-mosaic-details}

Вблизи фокальной поверхности сферического зеркала расположена мозаика из 109 ФЭУ (см. рис. \ref{pic:sphere-detector-optical-scheme}), собранных в приближённо гексагональную сетку. В центре расположен ФЭУ Hamamatsu R3886, характеризующийся б\`{о}льшим коэффициентом усиления, остальные -- ФЭУ 84-3. Hamamatsu использовался как референсный ФЭУ в процессе калибровки детектора \cite{SphereCalibration2016}.

\subsubsection{Рождение фотоэлектронов на фотокатоде ФЭУ}

Попадая на фотокатод ФЭУ, фотон с определённой вероятностью порождает фотоэлектрон, который затем под действием приложенной разности напряжений устремляется к первому диноду, на котором рождается ещё несколько электронов, и так далее, -- таким образом до анода доходит лавина фотоэлектронов, создающая заряд, достаточный для регистрации. Процесс фотоэффекта характеризуется функцией квантовой эффективности, которая характеризует вероятность, с которой фотон данной длины волны породит фотоэлектрон в системе.

Однако, поскольку отражающая способность снега в принятой модели (см. \ref{sect:snow-reflection}) не зависит от длины волны, спектр черенковских фотонов на фотокатоде оказывается таким же, как и в самом ливне. Поэтому для сокращения объёма данных при моделировании ШАЛ паспортная квантовая эффективность ФЭУ была заложена уже на этапе прослеживания ливня в атмосфере -- вместо спектра черенковского света сохранялась его свёртка с кривой квантовой эффективности, дающая ожидаемое число фотоэлектронов. Таким образом, в настоящей работе \textit{фотоны} ШАЛ всегда подразумеваются уже преобразованными в фотоэлектроны описанным способом, поэтому эти термины во многих случаях оказываются взаимозаменяемы, несмотря на фундаментальные физические различия.

\subsubsection{Лавинное усиление ФЭУ}
\label{sec:pmt-amplification-description}

Процесс развития электронной лавины на системе динодов ФЭУ -- принципиально статистический процесс. Это подтверждается прямыми лабораторными измерениями флуктуаций заряда, собранного на аноде Hamamatsu \cite[рис. 9]{SphereCalibration2016}, а также общими представлениями о механизме усиления: если динодная система насчитывает $N \approx 10$ динодов, а общий коэффициент размножения составляет в среднем $K \approx 10^6$, то средний коэффициент размножения на одном диноде будет составлять $\sqrt[N]{K} \approx 4$. Стохастический характер фотоэффекта приводит к тому, что истинный коэффициент умножения на каждом диноде будет иметь пуассоновское распределение с математическим ожиданием $\sqrt[N]{K}$. Эта простая модель позволяет получить распределение коэффициента усиления для ФЭУ84-3, для которого недоступны лабораторные измерения. Результаты этого моделирования и их следствия подробно обсуждаются в разделе \ref{sec:experimental-rir} (см. рис. \ref{pic:experimental-rir-params}).

Ясно, что статистический характер коэффициента усиления ФЭУ не играет большой роли при измерении больших потоков, так как происходит эффективное усреднение этой величины. Не важен он и при малых потоках, когда ФЭУ работает в режиме счёта фотонов и на осциллограмме наблюдаются отдельные хорошо разрешённые импульсы. Однако характерная интенсивность потока фотонов в эксперименте СФЕРА-2 такова, что режим работы попадет между этими двумя -- поток уже слишком велик, чтобы нельзя было разрешить отдельные фотоны, но ещё недостаточен, чтобы произошло эффективное усреднение. Именно этим обусловлена необходимость статистической деконволюции, описанной в главе \ref{chapt:bayesian-deconvolution}.

\subsubsection{Оцифровка импульса анодного тока}

Учитывая описанные условия, можно качественно охарактеризовать анодный ток ФЭУ. С одной стороны, постоянный поток фоновых фотонов -- преимущественно звёздного и зодиакального света, будут приводить к наличию стационарного тока. С другой стороны, фотоны ШАЛ, приходящие короткой сконцентрированным во времени пакетом, будут давать более и менее яркий импульс на этом фоне. Важный качественный результат лабораторных измерений состоит в том, что другие источники шума -- темновой ток или другие электронные шумы -- пренебрежимо малы по сравнению с сигналом от фоновых фотонов, поэтому в дальнейшем они учитываться не будут.

Подробное описание электроники детектора может быть найдено в работе \cite{SphereDetector2020}, здесь же ограничимся качественной картиной: постоянная и импульсная компоненты анодного тока разделяются RC-фильтром и записываются отдельными АЦП. Частота дискретизации для импульсной компоненты составляет $12.5~\text{нс}$, для постоянной -- показания АЦП сохраняются поминутно.

\subsubsection{Линейность}

Линейность системы регистрации света (связки ФЭУ и считывающей аппаратуры) была отдельно исследована по калибровочным кадрам. Интенсивность света в них намного выше даже самого яркого события ШАЛ, но даже в этих условиях амплитудная характеристика остаётся весьма близка к линейной, и может быть подвергнута дополнительной коррекции \cite[рис. 6]{SphereCalibration2016}. Поэтому в дальнейшем система регистрации света считается линейной.

	\chapter{Bayesian deconvolution}
\label{chapt:bayesian-deconvolution}

Early stages of the SPHERE-2 analysis included an attempt to construct the process of estimating the parameters of an EAS photons in each PMT (the photon count and the arrival time distribution) directly from the recorded signal, but it was not successful for a number of reasons. Instead, the idea arose to carry out a full deconvolution, that is, to extract information about the photon flux at PMTs input at each moment of time. This deconvolution should take into account the stochastic properties of the PMT. Similar problems are also considered in other areas, for example, in the processing of \cite{Rhode1993} spectral measurements and \cite{Wipf2013} images. In such problems, Bayesian statistics is fruitfully used, based on the interpretation of probability as a measure of information about the random variable or confidence in its value \cite{Gelman2013}.

\section{Problem setup}

The input signal is modeled as a set of $\delta$-functions offsetted in time. Time bins ($12.5~\mathrm{nsec}$ in the SPHERE-2 detector) provide natural time scale, and we assume that within each time bin, $\delta$-functions are distributed uniformly. We limit ourselves with $N$ time bins. The deconvolution will yield estimation for $n_i$, photon counts per $[i-1, i]$ time bin, where $i = 1 \ldots N$.

We assume that the PMT is a linear system with stochastic impulse response function $\tilde{h}(t)$ in the sense that each of the $\delta$-functions that make up the input signal is convolved with an independent sample $h(t) \sim\tilde{h}(t)$. This corresponds to the idea of fluctuations in the PMT amplification happening independently for each electron cascade. We will assume that any sample from $\tilde{h}(t)$ is causal, i.e. $\forall t < 0 \; \; \forall h \sim \tilde{h} \; \; h(t) = 0$, and finite in time, i.e. $\exists \tilde{L}$ such that $\forall t > \tilde{L} \; \; \tilde{h}(t) = 0$. We will call this random function, which gives a new sample for each input $\delta$-function, the system's randomized impulse response (RIR).

We pose the problem of statistical deconvolution using Bayesian terminology

\begin{quote}
	Given the randomized impulse response of the system $\tilde{h}(t)$ and the output signal $s_j, \; j = 1, \ldots, N + L$, find the posterior probability density functions for the values $n_i$, $i = 1, \ldots, N$.
\end{quote}

Note that, unlike regular deconvolution, we are not trying to estimate the full original signal (sum of $\delta$-functions), but only its integrated in each time bin.

Mathematically, we write

\begin{equation}
	P(\vec{n} | \vec{s}) = \frac{P(\vec{s} | \vec{n}) \, P(\vec{n})}{P(\vec{s})}
\end{equation}

Using \textit{uninformative} prior $P(\vec{n}) = Const$ (it can't be normalized, but we will use only proportionality, not an absolute value), denoting likelihood function $\mathcal{L}(\vec{s}, \vec{n}) = P(\vec{s} | \vec{n})$, we get

\begin{equation}
	\label{eq:bayes-theorem-adapted}
	P(\vec{n} | \vec{s}) \propto \mathcal{L}(\vec{s}, \vec{n})
\end{equation}

\section{Likelihood function}
\label{sec:naive-monte-carlo-likelihood}

PMT output signal $\vec{s}$ is a sample from some multivariate random variable. Denoting this variable $\vec{S}$, we can write

\begin{equation}
	S_j = \sum_{l=0}^{L} C(n_{j-l}, l)
	\label{eq:S-definition-as-random-variable}
\end{equation}

Here $C(n, l)$ is a random variable describing the contribution of $n$ $\delta$-functions with the \textit{delay} of $l$ bins. It is easy to see from the chosen bin indexing scheme and the conditions of causality and boundedness in time of the RIH that $l \in \left[0, L\right]$, since the contribution from $\delta$-functions from earlier bins is equal to zero.

The distribution of $C(n, l)$ can be studied with Monte-Carlo. Sampling it goes as follows. If we sample RIR $h_k(t) \sim \tilde{h}(t)$ and in-bin time $t_{inbin} \sim U(0, 1)$, and get a sample from $C(1, l) = h_k(l + 1 - t_inbin)$. Sampling from $C(n, l)$ is trivial, since the system is linear and we can just add $n$ independent samples from $C(1, l)$.

To calculate likelihood function, one needs to find a probability to find a particular $\vec{s}$ sampling from $\vec{S}$. Naive Monte-Carlo likelihood estimation requires sampling a large number of $\vec{n}$ and computing histogram in $\vec{s}$ space. This method is computationally unfeasible, because of the curse of dimensionality: in practice we are interested in signals at least 50 time bins long.

\subsection{Multivariate normal distribution approximation}
\label{sec:likelihood-as-multivar-normal}

We approximate likelihood function with the multivariate normal distribution. There is no formal proof for the applicability for such approximation, but we estimate that it is fairly close for input intensities as low as 4-5 $\delta$-functions per bin, which is the characteristic background intensity in SPHERE-2 detector. For partial univariate $s_j$ distribution, the probability density function is within 1-2\% of its normal approximation.

The general form of multivariate normal distribution is

\begin{equation}
	\label{eq:multivariate-normal-density}
	p(\vec{s}) = \left( (2 \pi)^{N+L} \, \det \Sigma \right)^{-1/2} \exp \left( - \frac{1}{2} (\vec{s} - \vec{\mu})^T \Sigma^{-1} (\vec{s} - \vec{\mu}) \right)
\end{equation}

Here, mean vector $\vec{\mu}$ and covariance matrix $\Sigma$ depend on input $\vec{n}$ and RIR $\tilde{h}(t)$. Specifically, $\vec{\mu} \equiv \mathbb{E} \vec{S}$ can be obtained from \ref{eq:S-definition-as-random-variable} by computing mean of left and right terms and using the fact that $\mathbb{E} C(n, l) = n \; \mathbb{E} C(1, l)$. The matrix $\Sigma$ can be expressed in terms of $C(n, l)$ autocovariance, which can be further expressed in terms of $C(1, l)$ autocovariance. The latter depend only on RIR and not on $\vec{n}$, and can be pre-calculated for computational effectiveness.

\begin{equation}
	\Sigma_{ij} = \cov(S_i, S_j) = \sum_{l=0}^{L - (i-j)} \cov(C(n_{i-l}, l), C(n_{i-l}, l + (i-j)))
\end{equation}


\section{Output signal error}

Multivariate normal approximation does not account for the error of measuring the output signal. The largest contribution to this error is ADC discretization. We assume that ADC floors its input, i.e. for each recorded $s_j$, the real input value was somewhere between $s_j$ and $s_j + \delta$ and the distribution is uniform. Such error can be accounted for by integrating the $\mathcal{L}_{\mathrm{exact}}(\vec{s}, \vec{n})$ defined by \ref{eq:multivariate-normal-density} over the $N$-dimensional cube with side $\delta$ and <<lower>> corner at $\vec{s}$:

\begin{equation}
	\label{eq:likelihood-with-error-uniform-error}
	\mathcal{L}(\vec{s}, \vec{n}) = \int_{s_1}^{s_1 + \delta} \int_{s_2}^{s_2 + \delta} \ldots \, \int_{s_{N}}^{s_N + \delta} \mathcal{L}_{\mathrm{exact}}(\vec{s}, \vec{n}) \, d\vec{s}
\end{equation}

To perform this integration numerically, we use the ready-made adaptive integration algorithm described in \cite{Genz1992}, implemented in the \texttt{scipy.stats} \cite{2020SciPy-NMeth} \texttt{Python} package.


\section{Posterior distribution sampling}
\label{sec:mcmc-sampling}

Having obtained the likelihood function $\mathcal{L}$, we can estimate $\vec{n}$ given the recorded output signal $\vec{s}$ and RIR $\tilde{h}(t)$. We want not only to find the optimal value of $\vec{n}$, but also characterize our confidence in it.

Instead of maximum likelihood method, where we would maximize the value of $\mathcal{L}(\vec{n}, \vec{s})$ with respect to $\vec{n}$, we use Marko Chain Monte Carlo sampling to draw a large sample from $P(\vec{n} | \vec{s}) \propto \mathcal{L}(\vec{n}, \vec{s})$. Using this sample of possible $\vec{n}$ values, we will then get the bets-fitting value for system's input, and its error.

The idea of the Markov Chain Monte Carlo (MCMC) family of methods \cite{Sharma2017} is to launch a Markov process (random walk) in the parameter space $\vec{n}$ with transition probability chosen in such a way that the <<trace>> of the walk will yield a sample from the target probability density functions. A common characteristic of this family is that they require only knowing only a ratio of probability density functions at different points in the parameter space, which is why in the expression (\ref{eq:bayes-theorem-adapted}) and in all subsequent calculations we were able to drop the marginal probability and the normalization of the prior distribution.

The particular method used in this work is affine-invariant MCMC \cite{Goodman2010}, implemented in \verb|Python| package \verb|emcee| \cite{ForemanMackey2016}.

\section{Deconvolution results}

A detailed description of the PMT connection and power supply circuits, the amplification circuit, as well as the characteristics of the used ADC converters can be found in \cite{SphereDetector2020}. For purposes of this work, we define detector's randomized impulse response as 

\begin{equation}
	\label{eq:experimental-rir}
	\tilde{h}(t) = C \, \tilde{C}_{PMT} \, h_I(t)
\end{equation}

\begin{figure}
	\centering
	\includegraphics[width=0.85\columnwidth]{experimental-ir-params}
	\caption{SPHERE-2 PMT's randomized impulse response. $h_I(t)$ (top panel) is the signal's shape in time domain, $\tilde{C}_{PMT}$ (bottom panel: PDF and CDF) --- dimensionless random PMT amplification coeffient.}
	\label{pic:experimental-rir-params}
\end{figure}

Where

\begin{enumerate}
	\item $h_I(t)$ --- shape of the impulse response, defined by time characteristics of the PMT and signal amplification circuit. Measured in the laboratory and plotted on the top panel of Fig. \ref{pic:experimental-rir-params}. Normalized to have integral 1.
	\item $\tilde{C}_{PMT}$ --- dimensionless random PMT amplification coefficient, normalized to have a mean value of 1. It's PDF and CDF are plotted on the bottom panel of Fig. \ref{pic:experimental-rir-params}.
	\item $C$ --- scale coefficient. It is measured during detector calibration and applied to signals earlier in data processing pipeline.
\end{enumerate}

We illustrate deconvolution procedure on toy input data: Poisson background with mean $\lambda=20$ photons per time bin and a <<signal>> photon packet: 3 bins with additional Poisson signal with $\lambda_{\mathrm{signal}}=40$. Fig. \ref{pic:bayesian-deconvolution-with-experimantal-rir-and-rounding} illustrates the system's input and output on the top panel and the deconvolution results on the bottom panel.


\begin{figure}
	\centering
	\includegraphics[width=0.85\columnwidth]{final-problem-and-solution}
	\caption{Bayesian deconvolution performed on synthetic toy input data. \textbf{Top panel:} toy input data $\vec{n}$ (blue) and corresponding system's output (convolution with randomized impulse response from fig. \ref{pic:experimental-rir-params}) $\vec{s}$, orange (X axis: <<Time, bins>>). \textbf{Bottom pannel}: same input data $\vec{n}$ (blue); rough $\vec{n}$ estimation from mean values, MCMC sampling starting point (red); deconvolution result, marginal posterior distributions in each time bin (green).}
	\label{pic:bayesian-deconvolution-with-experimantal-rir-and-rounding}
\end{figure}


	\chapter{Оценка параметров ШАЛ}

В результате проведения процедуры байесовской деконволюции получена безмодельная оценка потоков фотонов (в эквивалентных фотоэлектронах) на мозаике ФЭУ. Безмодельной она является в том смысле, что не зависит от детальных предположений о свойствах и источниках света, падающего на мозаику, кроме самых общих представлений. В этой главе на основе полученного результата, а также качественного представления о пространственно-временной структуре сигнала ШАЛ реконструируется функция пространственного распределения (ФПР) черенковского света. Наконец, на основе полученной ФПР делаются оценки параметров ШАЛ.

Стоит отметить, что методы, изложенные в этой главе, не являются новаторскими сами по себе, напротив, использованы уже хорошо изученные подходы. Основной целью является демонстрация того, как эти методы работают в контексте эксперимента СФЕРА-2 с учётом описанной процедуры деконволюции и неопределённостей, которые она порождает.

\section{Выделение сигнала ШАЛ и оценка значимости}

В процессе деконволюции не делается предположений о наличии или отсутствии сигнала ШАЛ в исследуемой области экспериментального кадра. Из моделирования ливня и оптической системы эксперимента известен общий вывод: фотоны ШАЛ достигают мозаики в виде \textit{пакетов} --- групп фотонов с приближённо нормальным распределением времён прихода. Каждый пакет можно описать тремя параметрами: $n_{EAS}$ --- суммарное число фотонов в пакете, $\mu_t$ --- среднее время прихода фотонов, $\sigma_t$ --- стандартное отклонение времён прихода. Ещё одним параметром является среднее число фоновых фотонов $\lambda_{n}$, однако, как показано в разделе \ref{sec:expdata-preparation-for-deconvolution}, оно известно из абсолютной калибровки сигнала.

\subsection{Выделение пакета фотонов ШАЛ}

\label{sec:signal-reconstruction}

Как и задача деконволюции (\ref{sec:bayesian-deconvolution-solution}), задача выделения сигнала ШАЛ может быть решена с помощью формализма байесовского вывода алгоритмом MCMC-сэмплирования. В качестве параметров модели используем $\Theta \equiv (n_{EAS}, \mu_t, \sigma_t)$, в качестве наблюдаемых данных --- полученную в результате деконволюции выборку значений $\vec{n}$.

Определим функцию правдоподобия для этой задачи: она должна давать вероятность того, что при фиксированном значении $\Theta$ будет получено наблюдаемое $\vec{n}$. То обстоятельство, что $\vec{n}$ измерен не прямо, а задан апостериорным распределением, легко учесть, просто усреднив значения функции правдоподобия по всем элементам этой выборки.

Представим $\vec{n}$ в виде суммы $\vec{n}_{EAS} + \vec{n}_{noise}$. Распределение случайного вектора $\vec{n}_{EAS}$ проще всего разыграть численно, генерируя выборки времён прихода фотонов объёмом $n_{EAS}$ из распределения $N(\mu_t, \sigma_t)$, и рассчитывая из них гистограмму в границах экспериментальных бинов. Для нахождения искомой функции правдоподобия остаётся вычислить вероятность того, что <<остаток>> фотоэлектронов $n_{noise}^{(i)}$ в каждом бине имеет пуассоновское распределение с $\lambda_{n}$:

\begin{equation}
	\mathcal{L}(\Theta) \equiv P(\vec{n} | n_{EAS}, \mu_t, \sigma_t) = \prod_{i} \frac{e^{-\lambda_n} \lambda_n^{n_{noise}^{(i)}}}{(n_{noise}^{(i)})!}
\end{equation}

Формула выше описывает значения правдоподобия при фиксированных $\vec{n}_{EAS}$ и  $\vec{n}$, поэтому для получения окончательного результата требуется усреднить значение $\mathcal{L}(\Theta)$ по соответствующим распределениям (флуктуациям гистограммы $\vec{n}_{EAS}$ и апостериорному распределению деконволюции).

В отличие от неинформативного априорного распределения, описанного для деконволюции в разделе \ref{sec:deconv-prior}, для $\Theta$ можно выбрать осмысленные априорные распределения из данных моделирования. Известно, что $n_{EAS}$ для интересующего нас диапазона энергий в $1$ - $100$ ПэВ имеет априорное распределение, экспоненциально спадающее от нуля с показателем $\sim 40$, а $\sigma_t$ --- $N(2.4, 1)$, обрезанное в нуле. Для $\mu_t$ было выбрано априорное распределение, равномерное на ширине окна деконволюции \footnote{В будущем при развитии методики можно уже на этапе поиска пакета <<угадывать>> его предполагаемое положение из приближённой оценки ориентации плоскости ливня по нескольким самым ярким каналам, и вносить эту информацию в априорное распределение $\mu_t$.}. Использование информативных априорных распределений помогает в процессе сэмплирования быстрее <<навестись>> на нужные области, например, сразу отбросить слишком широкие или многочисленные пакеты как маловероятные.

Техническая реализация MCMC-сэмплирования полностью аналогична описанной в разделе \ref{sec:mcmc-sampling}.

\begin{figure}
	\centering
	\includegraphics[width=\columnwidth]{signal-reconstruction-example}
	\caption{Выделение пакета фотонов ШАЛ из результатов байесовской деконволюции. Фиолетовыми кривыми представлены $10$ пакетов фотонов, соответствующих случайным элементам из апостериорной выборки $\Theta$ (см. текст). Видно, что пакеты группируются вблизи ожидаемого пика, но их параметры варьируются, учитывая неопределённость данных деконволюции.}
	\label{pic:signal-reconstruction-example}
\end{figure}


\subsection{Оценка значимости выделенного сигнала}

Одно из преимуществ полностью статистического подхода --- возможность использовать понятие значимости в процессе разделения сигнала ШАЛ и сигнала от фоновых фотонов. Широко принятый в байесовской статистике инструмент для этого --- байесовский информационный критерий (Bayesian information criterion, BIC), введённый Шварцем \cite{Schwarz1978} и представляющий собой в некотором смысле информационный критерий Акаике \cite{Akaike1974}, адаптированный для байесовского анализа. Суть его состоит в следующем: при наличии нескольких моделей, описывающих данные, их можно сравнить по количеству информации, которое теряется при замене данные на модель. Чем меньше потеря информации, тем меньше будет значение критерия, и тем лучше показывает себя модель. Вычисление проводится по формуле

\begin{equation}
	\mathrm{BIC} = k \ln n - 2 \ln \mathcal{L}_{max}
\end{equation}

Здесь $k$ --- число параметров модели, $n$ --- число элементов выборки данных, $\mathcal{L}_{max}$ --- максимальное значение функции правдоподобия для данной модели. Структура выражения указывает на важное качество критерия: чем больше число параметров модели $k$, тем больше требуемый прирост $\mathcal{L}_{max}$, это позволяет предотвратить переобучение модели с большим числом параметров.

Для определения значимости найденного сигнала ШАЛ нам нужно сравнить две модели: модель <<только шума>> ($n_{EAS} = 0$) вовсе без параметров ($k=0$), и модель <<шум + сигнал>>, описанную в предыдущем разделе, которая имеет $k=3$ параметра. По разнице $ \Delta \mathrm{BIC} = \mathrm{BIC}_{noise} - \mathrm{BIC}_{noise + EAS}$ можно судить о значимости восстановленного сигнала.

Если $\Delta \mathrm{BIC} \leq 0$, то модель только шума оказывается более состоятельной, и такой канал можно удалить из анализа. Если $0 < \Delta \mathrm{BIC} < 4$, то сигнал можно считать слабо значимым \cite{Kass1995}, на практике оказывается, что в эту область чаще всего попадают артефакты деконволюции или сильные флуктуации фона. Каналы, в которых $\Delta \mathrm{BIC} > 4$, принимались как достоверные (хотя и в этом случае иногда находятся артефакты, которые позже отсеиваются на этапе восстановления геометрии ливня).

На рис. \ref{pic:deconvolution-and-reconstruction} представлен пример сначала деконволюции, а затем восстановления параметров пакетов фотонов для экспериментального события. Оценка значимости сигналов использована, чтобы часть точек отсеить совсем, а часть --- пометить как <<сомнительные>>, однако видно, что в последних каналах присутствуют, по-видимому, артефакты деконволюции, которые дают сигналы высокой значимости, не ложащиеся в картину ливня. Их, впрочем, легко отделить по времени прихода, этот процесс будет описан в следующей главе. Тем не менее, наличие подобных ложных срабатываний затрудняет поиск истинных слабых сигналов ШАЛ, способы борьбы с ними разрабатываются --- в первую за счёт повышения вычислительной устойчивости процедуры деконволюции --- и будут применены в будущем.

\begin{figure}
	\centering
	\includegraphics[width=0.86\columnwidth]{deconvolution-and-reconstruction}
	\caption{Деконволюция и восстановление параметров пакетов для экспериментального события \#10675. Для параметров $n_{EAS}, \mu_t, \sigma_t$ ярким цветом показаны точки со значениями $\Delta \mathrm{BIC} > 4$, тусклым --- $\Delta \mathrm{BIC} \in (0, 4]$, значения с отрицательным $\Delta \mathrm{BIC}$ исключены, видно, что они соответствуют <<пустым>> областям кадра.}
	\label{pic:deconvolution-and-reconstruction}
\end{figure}


\section{Направление прихода ливня}

Стандартный метод оценки направления прихода (или, иначе, ориентации оси) ливня основан на представлении о том, что фронт черенковского света (а в случае с другими установками --- и заряженных частиц) с хорошей точностью является плоским. Поэтому, аппроксимируя плоскостью экспериментально измеренную зависимость времени прихода фронта $\bar{t}_{gnd}(x, y)$, мы сразу получаем оценку углов ориентации вектора нормали.

Для обработки данных эксперимента СФЕРА-2 требуется сделать дополнительный шаг: учесть время распространения света от снега до детектора. Это легко сделать, учитывая, что для каждого ФЭУ известен центр его поля зрения на поверхности $(x_i, y_i)$ (см. раздел \ref{sec:light-collection-from-surface} и в частности рис. \ref{pic:experimental-pmt-fov-example}), и отсюда, зная высоту подъёма установки $H$, получаем $\bar{t}_{i} = \mu_t^{(i)} - \frac{\sqrt{H^2 + x_i^2 + y_i^2}}{c}$.

Задача аппроксимации трёхмерных точек плоскостью решается линейным методом наименьших квадратов, однако сам по себе этот метод неустойчив к выбросам, а, как видно на рис. \ref{pic:deconvolution-and-reconstruction}, выбросы в данных $\mu_t$ присутствуют. Для решения этой проблемы была разработана методика итеративного фитирования: аппроксимируется набор точек, находится самая удалённая от плоскости точка, выбрасывается, новый набор аппроксимируется заново, и так далее. Процедура повторяется до тех пор, пока угол между векторами нормали в двух последовательных аппроксимациях не станет меньше заданного наперёд значения. Значение допустимого <<дрожания>> было положено равным $0.1^{\circ}$. Следует заметить, что это не ограничение на погрешность определения ориентации оси, но только на устойчивость оптимального значения этой ориентации относительно удаления <<наихудшей>> точки. Погрешность определения углов $\theta$ и $\phi$ получается естественным образом в процессе фитирования, зависит от числа точек и составляет в общем случае порядка нескольких градусов. На рис. \ref{pic:plane-reconstruction} показаны два примера этой процедуры для разных событий.


\begin{figure}
	\centering
	\includegraphics[width=0.45\columnwidth]{showe-plane-approximation-10675}
	\hfill
	\includegraphics[width=0.45\columnwidth]{showe-plane-approximation-10685}
	\caption{Восстановление ориентации ШАЛ для двух экспериментальных событий. Красными показаны точки, автоматически исключённые в процессе итеративного фитирования (см. текст).}
	\label{pic:plane-reconstruction}
\end{figure}

\subsection{Перспективы уточнения оценки}

Описанный метод является стандартным и устоявшимся, однако может быть уточнён с учётом индивидуальных особенностей эксперимента СФЕРА-2. В бакалаврской дипломной работе \cite{bachelorsthesis} был разработан метод уточнения оценки ориентации оси, рассматривающий ширину пакета фотонов как вторичный показатель. Идея метода основана на том, что для протяжённых полей зрения ФЭУ ожидается наличие зависимости $\sigma_t(x, y)$ --- для ФЭУ, через которые фронт ливня проходит первым, пакет фотонов будет \'{у}же, чем для расположенных на противоположной стороне мозаики. Этот <<геометрический эффект>> является следствием взаимной угла между падающим и рассеянным черенковским светом, и соответствующим сжатием или растяжением пакета во времени.

В настоящую работу такой анализ не вошёл, поскольку процедура требует отдельной модификации для работы с новыми данными, однако будет внесён в общий алгоритм обработки данных в дальнейшем.

\section{Положение оси ливня}

Для оценки положения оси ливня, то есть координат её пересечения с поверхностью наблюдения, можно использовать ряд методов; в этом разделе описан один из простейших, основанный только на предположении о том, что функция пространственного распределения черенковских фотонов монотонно убывает с расстоянием от оси ливня.

Из восстановленных параметров пакетов фотонов в каждом канале мы имеем зависимость оценки $n_{EAS}$, найденной для каждого канала в разделе \ref{sec:signal-reconstruction}, от координат на поверхности наблюдения. К этому моменту мы исключили из рассмотрения сигналы с низкой значимостью, а также сигналы, не вошедшие по времени прихода в восстановленную плоскость ливня (см. предыдущий раздел).

Сформулируем функцию правдоподобия, задающую вероятность того, что ось пересекает поверхность наблюдения в точке $(x_{ax}, y_{ax})$. Зная эти координаты, перенумеруем каналы по возрастанию расстояния от предполагаемого положения оси --- тогда функция правдоподобия будет равна вероятности того, что $\forall i, j: i < j$ будет выполняться $n_{EAS}^{(i)} > n_{EAS}^{(j)}$. Для простоты положим здесь и далее, что апостериорное распределение $n_{EAS}^{(i)}$ в $i$-том бине приближённо нормально, то есть полностью описывается средним $\mu_i$ и стандартным отклонением $\sigma_i$.

\begin{equation}
	\mathcal{L}(x_{ax}, y_{ax}) = \prod_i \prod_{j > i} \left( 1 - F_{N}(0, \mu_i - \mu_j, \, \sqrt{\sigma_i + \sigma_j}) \right)
\end{equation}

Здесь $F_{N}(x, \mu, \sigma) = \frac{1}{2} \left[ 1 + \mathrm{erf} \left( \frac{x - \mu}{\sqrt{2 \sigma^2}} \right) \right]$ --- функция распределения гауссовой случайной величины, её параметры $\mu_i - \mu_j$ и $\sqrt{\sigma_i + \sigma_j}$ соответствуют распределению разности двух нормалных случайных величин, $1 - F_N(0)$ даёт вероятность того, что разность случайных величин больше нуля.

Эту функцию правдоподобия можно максимизировать для поиска наиболее вероятного положения оси. Следует заметить, что, так как в основе $\mathcal{L}$ лежит только представление о монотонности ФПР, но не о характере зависимости потока от радиуса, область максимального правдоподобия будет иметь конечные размеры, в её пределах упорядочивание каналов по возрастанию расстояния от оси не будет изменяться, и функция правдоподобия будет иметь строго одинаковые значения. На практике оказалось, что область такого <<вырождения>> весьма мала, и метод пригоден по крайней мере для начальной оценки.

Более сложные методы, вероятно, могут использовать описанный метод как первое приближение, но в дальнейшем восстанавливать положение оси вместе с формой ФПР в едином фитировании. Переход к таким методам не представляет концептуальной сложности.

\section{Восстановление функции пространственного распределения черенковского света}

После определения ориентации и положения оси ливня в пространстве можно приступить к оценке пространственного распределения черенковского света ШАЛ. Форма ФПР черенковского света представляет существенный интерес, так как даёт способ оценки параметров ливня, слабо зависящий от модели ядерного взаимодействия. В частности, нормировка ФПР даёт информацию об энергии ливня, а форма (часто измеряемая показателем наклона, определяемого как отношение потоков на двух радиусах), предоставляет возможность для определения массы первичной частицы ШАЛ \cite{Patterson1983, Dawson1989, TOKUNO2008}.

Полный обзор и сравнение разных аппроксимаций черенковской ФПР остаётся за рамками настоящей работы, для демонстрации используем двухпараметрическую функцию, разработанную для аппроксимации данных детекторов черенковского света в эксперименте Тунка-25 \cite{Budnev2005}:

\begin{equation}
	\label{eq:tunka-25-ldf}
	\begin{gathered}
	Q(R) = Q_{kn} \cdot \begin{cases}
		\exp \left( \frac{(R_{kn} - R) \cdot (1 + 3/(R+3))}{R_0} \right) \text{ при } R < R_{kn} \\
		\left(\frac{R_{kn}}{R}\right)^b \text{ при } R \geq R_{kn} 
	\end{cases} \\
	R_0 = 10^{2.95 - 0.245 P}, \text{м}  \\
	R_{kn} = 155 - 13P, \text{м} \\
	b = 1.19 + 0/23 P
	\end{gathered}
\end{equation}

Параметр $Q_{kn}$ задаёт нормировку ФПР, а $P$ называется параметром наклона, он равен отношению $Q(100)/Q(200)$ и определяет форму пространственного распределения.

Также стоит отметить, что возможен другой подход, основанный не на аналитической аппроксимации ФПР, а на прямом сравнении экспериментальных данных с ФПР, полученной из Монте-Карло симуляции ливня. Предполагается, что такой метод должен давать наибольшую точность, хотя и требовать больше вычислительных ресурсов. Вне зависимости от способа получения функции $Q(x, y)$ и количества её варьируемых параметров, способ сравнения с экспериментальными данными остаётся тем же.

\subsection{Проекция полей зрения ФЭУ на плоскость ливня}

В разделе \ref{sec:light-collection-from-surface} описан процесс получения <<полей зрения>> ФЭУ --- распределений коэффициентов сбора $f^{(i)}(x_{gnd}, y_{gnd})$ по отражающей поверхности под установкой. Однако функция $Q(x_{shw}, y_{shw})$ задаётся в плоскости ливня, поэтому требуется спроецировать $f^{(i)}$ на неё же.

Для этого учтём, что плоскость ливня задаётся нормалью $(\theta_{shw}, \varphi_{shw})$ в системе земли и для определённости пересекает поверхность земли в начале координат. На плоскости ливня можно ввести систему координат с горизонтальной осью $x_{shw}$, направленной <<слева-направо>> с точки зрения движущегося ливня, и осью $y_{shw}$, лежащей в одной плоскости с вертикалью (<<снизу-вверх>> с точки зрения ливня). Тогда для произвольной точки на поверхности земли, выраженной полярными координатами $(r_{gnd}, \varphi_{gnd})$ из простых геометрических соображений найдём $x_{shw} = - r_{gnd} \, \sin (\varphi_{shw} - \varphi_{gnd})$, $y_{shw} = - r_{gnd} \, \cos (\varphi_{shw} - \varphi_{gnd}) \, \cos \theta_{shw}$.

Таким образом спроецированные поля зрения дают суммарное распределение чувствительности установки $f(x_{shw}, y_{shw})$ в плоскости фронта ливня. Пример такой проекции для ливня с зенитным углом падения $\approx 30^{\circ}$ приведён на рис. \ref{pic:projected-pmt-fov}. Заметим, что, так как функция $f$ является безразмерным коэффициентом сбора, нет необходимости учитывать преобразование элемента площади при проекции.

\begin{figure}
	\centering
	\includegraphics[width=\columnwidth]{projected-pmt-fov}
	\caption{Распределение чувствительности установки в плоскости фронта ливня для события \% 10675.}
	\label{pic:projected-pmt-fov}
\end{figure}

\subsection{Преобразование фотонов черенковского света ШАЛ в эквивалентные фотоэлектроны}

До сих пор мы, как указано в разделе \ref{sec:photon-to-phels-conversion}, работали с количеством эквивалентных фотоэлектронов, рождённых сигналом ШАЛ на фотокатоде ФЭУ. Для построения ФПР черенковского излучения требуется описать, как происходит пересчёт одной величины в другую.

Заметим, что искомая величина плотности черенковского света измеряется \textit{на единицу энергии}: $Q(x, y)$ [$\text{фотоны} \cdot \text{м}^{-2} \cdot \text{эВ}^{-1}$] \cite{Budnev2005}, так как в соответствии с формулой Франка-Тамма \cite{Tamm1939} $dN _{\gamma}/dE \approx Const$ в видимой части спектра. Тогда зная квантовую эффективность ФЭУ $\kappa(E)$, а точнее её среднее значение $\bar{\kappa}$ в диапазоне энергий $[E_{min}, E_{min} + \Delta E]$, можно перейти от $Q$ к флюенсу эквивалентных фотоэлектронов $n_{ph. el.} (x, y)$ [$\text{фотоэлектроны} \cdot \text{м}^{-2}$]:

\begin{equation}
	\Delta E \; \bar{\kappa} Q(x, y) = n_{ph. el.}
\end{equation}

Кривые квантовой эффективности для двух видов ФЭУ приведены на рис. \ref{pic:pmt-quanteffs}. Коэффициент $\Delta E \; \bar{\kappa}$ составляет $0.34~\text{эВ}$ для Hamamatsu R3886 и $0.39~\text{эВ}$ для ФЭУ84-3.

\begin{figure}[H]
	\centering
	\includegraphics[width=\columnwidth]{pmt-quantum-efficiences}
	\caption{Кривые квантовой эффективности для двух видов ФЭУ, использованных в эксперименте СФЕРА-2}
	\label{pic:pmt-quanteffs}
\end{figure}


\subsection{Неопределённость процесса сбора света с поверхности}

Учтём, что до сих пор мы получали апостериорную оценку на число фотоэлектронов, выбитых с фотокатода ФЭУ под действием фотонов ШАЛ. В то же время из модели ливня, учитывая коэффициенты сбора света с поверхности снега, мы можем получить только математической ожидание этой величины, но не её точное значение --- процесс сбора света сам является стохастическим. Хорошим приближением для него является пуассоновское распределение. Иначе говоря, если среднее число фотонов, ожидаемое к $i$-том канале, равно $\lambda^{(i)}$, то реально зарегистрировано будет $n^{(i)} \sim \mathrm{Poisson}(\lambda^{(i)})$. Таким образом нам необходимо, зная апостериорное (относительно процесса деконволюции, выделения пакета фотонов и т.д.) распределение $n$, нужно получить оценку $\lambda$ (индекс канала $i$ далее для краткости опущен).

Сначала рассмотрим случай, когда $n$ известно точно. Тогда теорема Байеса сразу даёт апостериорное распределение $\lambda$:

\begin{equation}
	P(\lambda | n) = P(n | \lambda) = \frac{e^{-\lambda} \lambda^n}{n!}
	\label{eq:lmbda-posterior-precise}
\end{equation}

Иначе говоря, плотность апостериорного распределения непрерывной величины $\lambda$ задаётся той же формулой, что и само распределение пуассона, но где $\lambda$ является аргументом, а $n$ --- параметром.

Теперь перейдём к случаю, где $n$ не известно точно, но имеет некоторое распределение. Учитывая, что $n$ может принимать только целые значения, представим выборку из апостериорного распределения в виде таблицы частотности: $\lbrace n_i, p_i \rbrace, i = 1, ..., k$, где $p_i$ --- вероятность, что $n$ имеет значение $n_i$. Тогда ясно, что апостериорное распределение $\lambda$ будет равно дискретной <<свёртке>> выражения (\ref{eq:lmbda-posterior-precise}) с распределением $n$:

\begin{equation}
P(\lambda) = \sum_{i=1}^{k} p_i \frac{e^{-\lambda} \lambda^{n_i}}{n_i !}
\end{equation}

На практике оказывается, что такая свёртка уширяет апостериорное распределение $n$ не более чем на $10$ -- $20 \%$, сдвиг математического ожидания из-за асимметрии распределения Пуассона пренебрежимо мал (менее $1 \%$).


\subsection{Аппроксимация ФПР}

Пользуясь результатами педыдущих двух разделов, запишем окончательное выражение для ожидаемого среднего числа фотоэлектронов от света ШАЛ, зарегистрированных в $i$-том ФЭУ.

\begin{equation}
	\int_{\infty} f^{(i)}(x, y)
	; \Delta E \; \bar{\kappa} \; Q(x, y) \; dx dy = \lambda_n^{(i)}
\end{equation}

Интегрирование можно провести численно по известным точкам распределения $f^{(i)}(x, y)$ в плоскости ливня. Используя данные апостериорного распределения $\lambda_n^{(i)}$, можно найти среднее значение и стандартное отклонение этой величины для использования в аппроксимации.

Визуализировать такое фитирование в традиционных координатах $R, Q(R)$ непросто, так как величина $Q(R)$ сворачивается с двумерной функцией распределения чувствительности для каждого ФЭУ. Приближённую качественную картину можно получить, если пренебречь протяжённостью полей зрения и считать, что они достаточно малы, чтобы можно было положить $Q(x, y) = Const = Q(x_c, y_c)$: $\Delta E \; \bar{\kappa} \; Q(x_c, y_c) \int_{\infty} f^{(i)}(x, y) \; dx dy = \lambda_n^{(i)}$. Это приближение позволяет приписать каждому измерительному каналу значение приближённое экспериментальное значение $Q$, которые можно затем нанести на график и  визуально сравнить с теоретической кривой.

Такая визуализация приведена на рис. \ref{pic:ldf-fit-example-1} и \ref{pic:ldf-fit-example-2}, однако следует подчеркнуть её неточность. В частности за счёт протяжённых полей зрения, эксперимент СФЕРА-2 чувствителен к черенковскому свету в приосевой области, поэтому точность и устойчивость фита заметно выше, чем может показаться по этим упрощённым графикам. Также точность повышается за счёт наложения полей зрения ФЭУ: в некоторых областях поверхности свет регистрируется сразу двумя или даже тремя ФЭУ, что эффективно уменьшает погрешности определения $\lambda_n$ в них.


\begin{figure}
	\centering
	\includegraphics[width=\columnwidth]{ldf-fit-10685}
	\caption{Упрощённая визуализация фитирования ФПР по данным эксперимента СФЕРА-2 (см. текст). Погрешность определения нормировки $Q_{kn}$ показана на графике коридором вокруг наиболее вероятной кривой. Инструментальная погрешность определения энергии вычислена по формуле (\ref{eq:E0-from-Q175}) с учётом погрешностей $Q_{kn}$ и $P$.}
	\label{pic:ldf-fit-example-1}
\end{figure}

\begin{figure}
	\centering
	\includegraphics[width=\columnwidth]{ldf-fit-10675}
	\caption{То же самое, что рис. \ref{pic:ldf-fit-example-1}, для другого экспериментального события.}
	\label{pic:ldf-fit-example-2}
\end{figure}


\subsection{Определение параметров ливня}

Используя описанную процедуру фитирования для оценки оптимального значения и неопределённости каждого из параметров ФПР, можно, наконец, перейти к оценке параметров первичной частицы ливня. За рамками данной работы остаются вопросы связи энергии и массы первичной частицы, а также глубины максимума, с наблюдаемыми характеристиками ФПР. На этом этапе необходимо учитывать модельные неопределённости, то есть статистический характер зависимости параметров ФПР от исследуемых характеристик ливня.

В этом разделе ограничимся простым применением формулы для связи $E_0$ и $Q(R)$ \cite{Budnev2005}:

\begin{equation}
	\label{eq:E0-from-Q175}
	E \; [\text{ТэВ}] = 400 \cdot Q(175)^{0.95}
\end{equation}

Чтобы определить неопределённость $Q_{175}$, и как следствие $E$ с учётом оцененных неопределённостей параметров $Q_{kn}$ и $P$ сэмплируем каждый из них и вычислим среднее и стандартное отклонение по декартову произведению выборок (как видно из выражения (\ref{eq:tunka-25-ldf}), ФПР нетривиальным образом зависит от $P$).

Полученные энергии с соответствующими погрешностями приведены на рис. \ref{pic:ldf-fit-example-1} и \ref{pic:ldf-fit-example-2} вместе с параметрами ФПР. Погрешность определяется для каждого события отдельно, но в общем по двум примерам с различными энергиями и положениями оси относительно детектора её можно оценить в $5$ -- $10 \%$. Детальное исследование систематической инструментальной погрешности (в том числе её зависимость от энергии и положения ливня, высоты и ориентации установки) будет предметом дальнейших изысканий с использованием набранной статистики событий.

Также следует качественно сравнить полученные величины с модельными погрешностями, которые могут быть найдены, например, в работе \cite[табл. 1]{Anokhina2007}. Для ливней с энергией $1$ -- $10$ ПэВ флуктуации $Q(150)$, например, оцениваются в $2$ -- $10 \%$, уменьшаясь с энергией и массой первичной частицы. Для событий с энергиями $\approx 80$ и $\approx 206$ ПэВ эти модельные неопределённости будут ещё меньше --- следовательно, особенно важным становится именно точное определение инструментальной погрешности.


	% \chapter*{ЗАКЛЮЧЕНИЕ}
\addcontentsline{toc}{chapter}{ЗАКЛЮЧЕНИЕ}

В настоящей работе развиваются методы обработки данных эксперимента СФЕРА-2, нацеленные на восстановление детальной структуры ШАЛ. Основное направление развития состоит в построении более глубокой и детальной модели детектора и оценки инструментальных неопределённостей, сопряжённых с процессом регистрации черенковского света ШАЛ. Предложен новый, более модульный по сравнению с предшествующими работами, принцип построения процедуры анализа, призванный обеспечить её большую прозрачность и дать инструменты для перекрёстной проверки всех промежуточных выводов.

Для достижения поставленных целей реализован оригинальный метод деконволюции на основе байесовской статистики и метода Монте-Карло с марковскими цепями, приведены обоснования его корректности и примеры применения. Байесовский формализм для описания восстановленных величин с помощью их апостериорных распределений оказался удобным и во всём последующем анализе, и его применение предлагается расширять.

Результаты разработанных алгоритмов деконволюции и выделения сигнала ШАЛ следует считать предварительными: ещё предстоит провести тщательную проверку, оптимизацию, детальное сравнение с результатами предыдущих методов. Алгоритмы оценки параметров ливня и первичной частицы, приведённые в работе, также являются предварительными и служат в большей степени для проверки концепции.

Однако уже эти новые результаты представляются качественно верными. В частности, оценка относительной инструментальной погрешности определения энергии первичной частицы ШАЛ в $5$ -- $10 \%$ согласуется с ожидаемым значением, в частности, с систематическими погрешностями других экспериментов по регистрации ШАЛ. В то же время, так как это значение получается пособытийно, несколько примеров, приведённых в работе, несомненно, не могут дать полной картины.

В духе принципов открытой науки весь анализ проведён с помощью свободных программных инструментов с открытым исходным кодом: на языке \verb|Python| с использованием библиотек NumPy, SciPy, emcee. Все оригинальные коды, с помощью которых получены приведённые результаты, доступны в публичном git-репозитории \footfullcite{githubRepo}.

	
	% \phantomsection
	% \addcontentsline{toc}{chapter}{СПИСОК ИСПОЛЬЗОВАННЫХ ИСТОЧНИКОВ}
	\printbibliography[title={Bibliography}]
	
\end{document}
