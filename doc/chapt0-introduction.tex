\chapter*{Введение}
\addcontentsline{toc}{chapter}{ВВЕДЕНИЕ}

Космические лучи (КЛ) -- заряженные частицы высоких энергий, движущиеся в космическом пространстве -- являются важным фактором динамики космических систем и давно изучаются целым рядом экспериментальных методов. Спектр космических лучей простирается на диапазоне почти $12$ порядков по энергии (от $10^{9}$ до $10^{21}$ эВ) и имеет ряд особенностей, которые несут информацию об их происхождении и условиях распространения в Солнечной системе, Галактике и Метагалактике. Направления прихода ливней распределены практически изотропно для большей части спектра, что также даёт важную информацию об условиях их распространения, а в область сверхвысоких энергий ведутся поиски анизотропии, которая должна подсказать распределение источников этих частиц. В составе космических лучей на сравнительно небольших энергиях преобладают протоны, доли прочих химических элементов также хорошо изучены, однако для высоких энергий однозначных данных нет, различные эксперименты дают противоречивые оценки состава. Состав космических лучей, связанный с их временем жизни, тоже весьма ценен для задач описания распределения источников космических лучей и магнитных полей, определяющих их распространение.

В области энергий $> 10^{15}$ эВ фактически единственным способом изучения космических лучей является непрямой метод регистрации каскадов вторичных частиц, порождаемых КЛ в атмосфере -- широких атмосферных ливней (ШАЛ). Этот метод ставит земную атмосферу на службу экспериментальной технике, превращая её в гигантский калориметр. За последние $50$ лет разработано много методов регистрации ШАЛ по разным их компонентам или эффектам, который они оказывают на атмосферу. Настоящая работа посвящена обработке данных эксперимента СФЕРА-2, который нацелен на регистрацию ШАЛ сравнительно слабо изученным методом отражённого черенковского света.

Целью работы является развитие и продолжение уже созданных методов обработки данных, но обогащение их более глубоким пониманием принципов работы детектора СФЕРА-2, достигнутым в последние годы. В частности, фокус стоит на количественном измерении инструментальных неопределённостей, вносимых в процессе измерения, которые в предыдущих публикациях описывались лишь в общих чертах. Для этого была поставлена задача моделирования детектора как стохастической системы и разработки адекватных этому методов оценки регистрируемых параметров.

С методической точки зрения в данной работе предлагается более модульный по сравнению с предыдущими публикациями подход, основанный на последовательном применении ряда сравнительно независимых процедур реконструкции, вместо единой модели типа <<чёрный ящик>>. С точки зрения математического аппарата, в задаче описания детектора, было найдено плодотворное применение понятиям байесовской статистики и соответствующим численным алгоритмам.
