\chapter*{Introduction}

Cosmic rays (CR) are highly energetic charged particles that traverse space. They play a crucial role in understanding the dynamics of space systems. Their energy spectrum spans $12$ orders of magnitude ($10^9$ to $10^{21}$ eV), providing insights into their origin and propagation within the Solar System, Galaxy, and Metagalaxy. Above $10^{15}$ eV, indirect methods are employed to study cosmic rays through the detection of extensive air showers (EAS), utilizing Earth's atmosphere as a giant calorimeter.

The SPHERE-2 experiment focuses on detecting EAS using the relatively novel method of reflected Cherenkov light. This work enhances the experiment's data processing by incorporating a deeper understanding of the SPHERE-2 detector gained in recent years. The key improvement lies in modeling the detector as a stochastic system, enabling accurate estimation of instrumental uncertainties. Bayesian statistics serves as the primary mathematical framework for the analysis.
