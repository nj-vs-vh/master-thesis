\chapter*{Введение}
\addcontentsline{toc}{chapter}{ВВЕДЕНИЕ}

Космические лучи (КЛ) --- заряженные частицы высоких энергий, движущиеся в космическом пространстве --- являются важным фактором динамики космических систем и давно изучаются с помощью многих экспериментальных методов. Энергетический спектр космических лучей простирается почти на $12$ порядков --- от $10^{9}$ до $10^{21}$ эВ --- и имеет ряд особенностей, которые несут информацию об их происхождении и условиях распространения в Солнечной системе, Галактике и Метагалактике. Направления прихода ливней распределены практически изотропно для большей части спектра, что также даёт важную информацию об условиях их распространения, а в области сверхвысоких энергий ведутся поиски анизотропии, которая должна указать на распределение источников этих частиц. В составе космических лучей на сравнительно небольших энергиях преобладают протоны, меньшие доли прочих химических элементов также хорошо изучены, однако для высоких энергий нет однозначных данных, различные эксперименты дают противоречивые оценки состава. Состав космических лучей, связанный с их временем жизни, тоже весьма ценен для ответа на вопросы о распределении источников космических лучей и о крупномасштабных магнитных полях, определяющих их распространение.

В области энергий больше $10^{15}$ эВ фактически единственным способом экспериментального изучения космических лучей является непрямой метод регистрации каскадов вторичных частиц, порождаемых КЛ в атмосфере --- широких атмосферных ливней (ШАЛ). Этот метод ставит земную атмосферу на службу экспериментальной технике, превращая её в гигантский калориметр. За последние $50$ лет разработано много методов регистрации ШАЛ по разным их компонентам и эффектам, который они оказывают на атмосферу. Настоящая работа посвящена обработке данных эксперимента СФЕРА-2, который нацелен на регистрацию ШАЛ методом отражённого черенковского света --- сравнительно новым и мало изученным.

Целью работы является развитие и продолжение уже созданных методов обработки данных, но обогащение их более глубоким пониманием принципов работы детектора СФЕРА-2, достигнутым в последние годы. В частности, акцент стоит на количественном измерении инструментальных неопределённостей, вносимых в процессе измерения, которые в предыдущих публикациях описывались лишь в общих чертах. Для достижения этой цели была поставлена задача моделирования детектора как стохастической системы и разработки адекватных этому методов оценки параметров ливня.

С методической точки зрения в данной работе предлагается более модульный по сравнению с предыдущими публикациями подход, основанный на последовательном применении ряда сравнительно независимых процедур реконструкции, вместо единой модели типа <<чёрный ящик>>. С точки зрения используемого математического аппарата, в задаче описания детектора, было найдено плодотворное применение понятиям байесовской статистики и соответствующим численным алгоритмам.
