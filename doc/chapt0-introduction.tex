\chapter*{Introduction}

Cosmic rays (CR) are highly energetic charged particles moving through space. They are an important factor in the dynamics of space systems and have long been studied using many experimental methods. The energy spectrum of cosmic rays spans almost $12$ orders of magnitude from $10^{9}$ to $10^{21}$ eV and has several features that carry information about their origin and propagation mechanisms in the Solar System, the Galaxy, and the Metagalaxy. The arrival directions of showers are distributed almost isotropically for most of the spectrum, which also provides important information about the conditions of their propagation. In the ultrahigh energy region, the search for anisotropy is ongoing, which should reflect the distribution of cosmic ray sources.

In the energy range above $10^{15}$ eV the only way to study cosmic rays is indirectly, by detecting cascades of secondary particles generated by CR in the atmosphere --- extensive air showers (EAS). This method puts the Earth's atmosphere at the service of experimental technology, essentially turning it into a giant calorimeter. Over the past $50$ years, many methods have been developed to detect EAS observing their various components and the effects they produce in the atmosphere. The present work is devoted to the processing of data from the SPHERE-2 experiment, which is aimed at detecting EAS by the method of reflected Cherenkov light, which is relatively new and little studied.

This work develops and extends the established data processing method, enriching it with a deeper understanding of the SPHERE-2 detector achieved in recent years. In particular, the emphasis is on calculating instrumental uncertainties introduced during the measurement, which were described in the previous work only approximately. To achieve this goal, the detector was modeled as a stochastic system, and shower parameter estimation methods were updated to reflect that.

From a methodological point of view, this work employs a more modular or pipeline approach compared to previous publications, based on the sequential application of a number of relatively independent reconstruction procedures, instead of a single <<black box>> model. The mathematical apparatus for this work relies on the concepts of Bayesian statistics and the corresponding numerical algorithms.