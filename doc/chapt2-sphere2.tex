\chapter{Эксперимент СФЕРА-2}

В этой главе описывается идея, мотивация и краткая история метода регистрации отражённого черенковского света, который развивается в эксперименте СФЕРА-2; обозреваются проведённые измерения и полученные на сегодняшний день результаты, обосновывается необходимость и основные направление дальнейшей работы; наконец, приводится подробная качественная модель процесса регистрации ШАЛ детектором СФЕРА-2.

\section{Метод регистрации отражённого черенковского света ШАЛ}

Черенковский свет ШАЛ является важным инструментом исследования ШАЛ, в первую очередь из-за относительно слабой зависимости его полного потока от модели ядерного взаимодействия при высоких энергиях. Традиционный способ прямой регистрации черенковского света наземными детекторами геометрически аналогичен наземной регистрации заряженных частиц: также используется массив детекторов, распределённых по большой площади, которые дают точечные измерения плотности потока.

Однако, в отличие от заряженных частиц, черенковский свет оптического диапазона может весьма эффективно рассеиваться поверхностью Земли и наблюдаться уже отражённым. Метод регистрации этого отражённого света первым предложил академик А. Е. Чудаков \cite{Чудаков1972}. Одной из первых практических реализаций подобной идеи стали измерения Д. Наварры \cite{Navarra1981}. Позже Р. А. Антоновым была предложена идея установки на основе оптической схемы Шмидта: с чувствительными элементами вблизи фокальной поверхности сферического зеркала. Первые эксперименты были проведены на базе Тянь-Шаньской установки по регистрации ШАЛ, дальнейшее развитие этой методики велось в экспериментах СФЕРА-1 и СФЕРА-2 \cite{Antonov1997, Antonov2001}. Подробное описание этого эксперимента приведено в разделе \ref{sec:sphere-2-model}.

Преимущества метода, помимо общих достоинств методики регистрации черенковского света, включают возможность достичь существенной площади регистрации компактным детектором, а также доступ (хотя и косвенный) к черенковскому свету из приосевой области ливня, достигаемый за счёт протяжённых полей зрения отдельных чувствительных элементов прибора. Результаты предварительного моделирования подтверждают эти ожидания \cite{Anokhina2007}.


\section{Методы обработки данных и предварительные результаты}

Данные эксперимента СФЕРА-2 за 2011-2013 годы анализировались для построения дифференциального энергетического спектра и оценки доли лёгких ядер в зависимости от энергии \cite{Antonov2013, Chernov2017-ICRC}. Для данных за 2013 год результаты этих исследований представлены на рис. \ref{pic:sphere-previous-results}.

\begin{figure}
	\centering
	\includegraphics[width=0.75\columnwidth]{sphere-2013-spectrum}
	\\
	\includegraphics[width=0.75\columnwidth]{sphere-2013-composition}
	\caption{Предварительные результаты измерения дифференциального энергетического спектра и доли лёгких ядер по данным эксперимента СФЕРА-2. Графики из работы \cite[рис. 8 и рис. 10]{Chernov2017-ICRC}}
	\label{pic:sphere-previous-results}
\end{figure}

Идеи и физические представления о ШАЛ и об экспериментальной установке, лежащих в основе предварительного анализа, наследуется и развивается в настоящей работе. С другой стороны, между ними можно провести методическое различение в общем подходе к моделированию. В предшествующих работах доминирующим был холистический подход, стремящийся построить замкнутую модель типа <<чёрного ящика>> для изучаемого явления и его процессов эксперименталной регистрации \cite{Chernov2015}. На вход такой модели подаются физические параметры первичной частицы (энергия, масса, направление движения), а также ряд параметров установки и среды в момент регистрации, а на выходе получается <<модельный>> кадр события. Оценки интересующих нас параметров первичной частицы предполагается давать, сравнивая экспериментальные кадры с модельными. Иллюстрацией такой методической замкнутости модели может служить процесс построения ФПР, описанный в работе \cite{Sphere2015} и проиллюстрированный на рис. \ref{pic:sphere-previous-ldf}: <<единицы кода>>, в которых строится распределение, не являются физическими, но относятся только к эффекту от ливня в установке (понятие единиц кода обсуждается подробнее в разделе \ref{sec:expdata-preparation-for-deconvolution}).

\begin{figure}
	\centering
	\includegraphics[width=0.6\columnwidth]{sphere-2013-ldf-reconstruction}
	\caption{Иллюстрация процесса восстановления ФПР черенковского света для предварительного анализа данных СФЕРА-2. Значения, отложенные по вертикали, измеряются в единицах кода, характеризующих конечный сигнал, усиленный и оцифрованный электроникой детектора (см. текст и раздел \ref{sec:expdata-preparation-for-deconvolution}). График из работы \cite[рис. 27]{Sphere2015}}
	\label{pic:sphere-previous-ldf}
\end{figure}

В настоящей работе предпринята попытка построения более модульного подхода, требующая <<расцепить>> единую модель на отдельные модели ливня и детектора (а модель детектора --- далее на модели его частей). Точкой их смыкания должна быть физическая характеристика ливня, с одной стороны, естественным образом получаемая из модели развития в атмосфере, а с другой --- оцениваемая из экспериментальных данных. В качестве такой характеристики представляется возможным выбрать распределение плотности черенковского света в плоскости фронта ливня, хотя в общем случае этот выбор является предметом соглашения.


\section{Качественная модель работы детектора}
\label{sec:sphere-2-model}

Опишем на качественном уровне теоретические представления, которые лежат в основе интерпретации данных СФЕРА-2. Для этого проследим, что происходит от развития каскада в атмосфере и до получения файлов с экспериментальными данными.

\subsection{Развитие ШАЛ в атмосфере}

Сколько-нибудь полный обзор теории ШАЛ, и даже черенковского света ШАЛ в отдельности, лежит за рамками данной работы, поэтому изложим только самые общие представления. Широкий атмосферный ливень --- каскад вторичных частиц, вызванный взаимодействием первичной частицы большой энергии с атмосферой --- развивается в виде тонкого диска заряженных частиц: адронов, мезонов, лептонов, гамма-квантов. Черенковский свет --- одно из сопутствующих излучений ШАЛ, возникающее при движении заряженных частиц со скоростью, превышающую скорость света в среде распространения. Черенковский свет ШАЛ также распространяется в виде тонкого, до нескольких метров в толщину, диска, ориентированного перпендикулярно оси ливня. Для практических задач моделирование процессов развития ШАЛ проводится численно, например, с помощью программы CORSIKA \cite{CORSIKA-report}.

Уже на этапе моделирования ШАЛ закладывается ряд неопределённостей реконструкции параметров первичной частицы. Во-первых, модели ядерного взаимодействия при энергиях больше примерно $10^{16}~\text{эВ}$ не проверены экспериментально, являясь, в сущности, экстраполяциями, и разные такие модели приводят к несколько разным картинам развития ливня для одной и той же первичной частицы. Во-вторых, на развитие ливня сильно влияет состояние атмосферы, которое тоже известно лишь с некоторой погрешностью. Наконец, развитие ШАЛ --- принципиально стохастический процесс, что приводит к статистической неопределённости любой реконструкции. Эти неопределённости будем в дальнейшем называть \textit{модельными}, так как они напрямую связаны с моделью описываемого явления, в противоположность \textit{инструментальным} неопределённостям, связанным с процессом измерения характеристик ливня установкой.

\subsection{Отражение света от снега}
\label{sect:snow-reflection}

Эксперименты, использующие отражённый черенковский свет, предъявляют особые требования к земной поверхности на уровне наблюдения --- она должна служить <<экраном>>, который бы равномерно и предсказуемо рассеивал свет. В случае СФЕРЫ-2 таким экраном служит заснеженный лёд озера Байкал.

Рассеивающие свойства снега могут существенно повлиять на работу эксперимента, поэтому для измерения и мониторинга этих условий были приложены определённые усилия. Помимо общих данных о свойствах снега \cite{Warren1982} были проведены прямые измерения коэффициента отражения в зависимости от угла \cite[рис.~11]{Sphere2015}. В результате них был сделан вывод, что зависимость хорошо согласуется с законом рассеяния Ламберта для идеальной диффузной поверхности: \textit{яркость} рассеянного света не зависит от угла, то есть интенсивность имеет чисто геометрическую зависимость $I \propto \cos \theta_n$ \cite{Antonov2019}. Альбедо $a$ --- отношение падающего и отражённого потоков --- было принято независимым от длины волны и равным $0.95$.

\subsection{Сбор отражённого света}
\label{sec:light-collection-from-surface}

Установка СФЕРА-2 поднималась аэростатом на высоту $400$ -- $900~\text{м}$, для сбора света с поверхности использовалась оптическая система из диафрагмы, сферического зеркала и мозаики ФЭУ, которая схематично изображена на рис. \ref{pic:sphere-detector-optical-scheme}. В результате каждый из ФЭУ обозревал область на поверхности диаметром $10$ -- $50~\text{м}$ в зависимости от высоты подъёма.

\begin{figure}
	\centering
	\includegraphics[width=\columnwidth]{optical_scheme}
	\caption{Оптическая схема детектора СФЕРА-2}
	\label{pic:sphere-detector-optical-scheme}
\end{figure}

Качественно оценим коэффициент сбора света: пусть на участок поверхности в окрестности точки $(x, y)$ в системе координат с центром в проекции детектора падает $\delta N_{gr}$ фотонов, и пусть горизонтально ориентированная диафрагма радиусом $R_{d}$, поднята на высоту $H$. Обозначая угол рассеяния света от нормали $\theta_n$, можно оценить число фотонов, которое достигнет диафрагмы из результатов предыдущего раздела и геометрических соображений

\begin{equation}
	\delta N_{d} \approx \frac{R_d^2 \cos \theta_n}{H^2 + x^2 + y^2} K \cos \theta_n \delta N_{gr}
\end{equation}

Из этого простого расчёта при характерных значениях $H = 600~\text{м}$, $R_d \approx 0.465~\text{м}$ получается оценка общего коэффициента сбора установки: $(0.5 \div 1) \cdot 10^{-6}$.

Для расчёта количества фотонов, достигающего каждого отдельного ФЭУ, требуется численное моделирование распространения света внутри оптической системы установки: отражения от зеркал и стёкол ФЭУ, поглощения тыльной стороной мозаики и элементами конструкции. Результатом такого моделирования является коэффициент сбора света как функция координат на поверхности, или \textit{чувствительность} каждого ФЭУ $f^{(i)}(x, y)$. Более строго, если флюенс черенковских фотонов равен $n(x, y)$, то число ожидаемое число фотонов, попавших в $i$-тый ФЭУ, будет равно $\int_{\infty} f^{(i)}(x, y) n(x, y) dx dy$.

Функции $f^{(i)}$ имеют форму пятен, соответствующих полям зрения ФЭУ на поверхности льда. Пример нескольких таких пятен, а также суммарной чувствительности $\sum_{i} f^{(i)}(x, y)$ всей мозаики для одного из экспериментальных событий изображён на рис. \ref{pic:experimental-pmt-fov-example}. Ясно, что поля зрения зависят от высоты подъёма и ориентации установки, поэтому при расчёте учитывается данные GPS и инклинометра соответственно.

\begin{figure}
	\centering
	\includegraphics[width=\columnwidth]{experimental-pmt-fov-example}
	\caption{Пример моделирования сбора света с поверхности установкой СФЕРА-2 для экспериментального события \#10699. Показаны чувствительности трёх отдельных ФЭУ и суммарная чувствительность всей мозаики. Учтены данные о высоте подъёма и наклоне установки, поэтому картина чувствительности сдвинута относительно проекции установки и слегка вытянута.}
	\label{pic:experimental-pmt-fov-example}
\end{figure}

\subsection{Регистрация собранного света}

Финальный этап процесса регистрации события --- преобразование интенсивности фотонов, попавших на каждый ФЭУ, в оцифрованные данные --- включает несколько подэтапов, которые мы изложим особенно подробно, поскольку именно они обуславливают многие исследуемые в настоящей работе эффекты.

\subsubsection{Мозаика ФЭУ}
\label{sec:pmt-mosaic-details}

Вблизи фокальной поверхности сферического зеркала расположена мозаика из 109 ФЭУ (см. рис. \ref{pic:sphere-detector-optical-scheme}), собранных в приближённо гексагональную сетку. В центре расположен ФЭУ Hamamatsu R3886, характеризующийся б\`{о}льшим коэффициентом усиления и площадью фотокатода, остальные --- ФЭУ 84-3. Hamamatsu использовался как референсный ФЭУ в процессе калибровки детектора \cite{SphereCalibration2016}.

\subsubsection{Рождение фотоэлектронов на фотокатоде ФЭУ}

\label{sec:photon-to-phels-conversion}

Попадая на фотокатод ФЭУ, фотон с определённой вероятностью порождает фотоэлектрон, который затем под действием приложенной разности напряжений устремляется к первому диноду, на котором рождается ещё несколько электронов, и так далее. Таким образом до анода доходит лавина фотоэлектронов, создающая заряд, достаточный для регистрации. Процесс выбивания фотоэлектронов с катода характеризуется функцией квантовой эффективности, которая характеризует вероятность, с которой фотон данной длины волны породит фотоэлектрон в системе.

Однако, поскольку отражающая способность снега в принятой модели (см. раздел \ref{sect:snow-reflection}) не зависит от длины волны, спектр черенковских фотонов на фотокатоде оказывается таким же, как и в самом ливне. Поэтому для сокращения объёма данных при моделировании ШАЛ паспортная квантовая эффективность ФЭУ была заложена уже на этапе прослеживания ливня в атмосфере: вместо спектра черенковского света сохранялась его свёртка с кривой квантовой эффективности, дающая ожидаемое число фотоэлектронов. Таким образом, в настоящей работе \textit{фотоны} ШАЛ чаще всего (если не указано обратное) подразумеваются уже преобразованными в фотоэлектроны описанным способом, эти величины в большинстве случаях оказываются взаимозаменяемы, несмотря на фундаментальные физические различия.


\subsubsection{Усиление сигнала ФЭУ}
\label{sec:pmt-amplification-description}

Процесс развития электронной лавины на системе динодов ФЭУ является принципиально стохастическим. Это подтверждается прямыми лабораторными измерениями флуктуаций заряда, собранного на аноде эталонного ФЭУ \cite[рис. 9]{SphereCalibration2016}, но также и общими представлениями о механизме усиления: если динодная система насчитывает $N \approx 10$ динодов, а общий коэффициент размножения составляет в среднем $K \approx 10^6$, то среднее размножение на одном диноде будет составлять $\sqrt[N]{K} \approx 4$. Стохастический характер этого процесса означает, что истинный коэффициент умножения на каждом диноде будет иметь приближённо пуассоновское распределение с математическим ожиданием $\sqrt[N]{K}$. Из этих представлений, а также учитывая некоторые специальные особенности ФЭУ (выделенный по множественности первый динод, шанс не попасть в динод вообще) можно получить распределение коэффициента усиления для ФЭУ84-3, для которого недоступны лабораторные измерения. Результаты этого моделирования и их следствия подробно обсуждаются позже в разделе \ref{sec:experimental-rir} (см. рис. \ref{pic:experimental-rir-params}).

Ясно, что стохастический характер коэффициента усиления ФЭУ не играет большой роли при измерении больших потоков, так как происходит эффективное усреднение этой величины. Не важен он и при малых потоках, когда ФЭУ работает в режиме счёта фотонов и на осциллограмме можно наблюдать отдельные хорошо разрешённые импульсы. Однако характерная интенсивность потока фотонов и временные характеристики системы регистрации в эксперименте СФЕРА-2 таковы, что режим работы оказывается как раз в промежутке между этими двумя случаями --- поток слишком велик, чтобы нельзя было разрешить отдельные фотоны, но не достаточен для того, чтобы произошло эффективное усреднение. Именно этим обусловлена необходимость статистической деконволюции, описанной в главе \ref{chapt:bayesian-deconvolution}.

\subsubsection{Оцифровка импульса анодного тока}

Учитывая описанные условия, можно охарактеризовать анодный ток ФЭУ. С одной стороны, постоянный поток фоновых фотонов --- преимущественно атмосферного свечения, звёздного и зодиакального света, будут приводить к наличию постоянной компоненты тока. С другой стороны, фотоны ШАЛ, приходящие коротким сконцентрированным во времени пакетом, будут выделяться импульсом на этом фоне. Важный качественный результат лабораторных измерений состоит в том, что другие, нефотонные источники шума в системе (например, темновой ток ФЭУ) пренебрежимо малы, поэтому в дальнейшем они учитываться не будут.

Подробное описание электроники детектора может быть найдено в работе \cite{SphereDetector2020}, здесь же ограничимся качественной картиной. С помощью RC-фильтра постоянная и импульсная компоненты анодного тока разделяются и записываются отдельными АЦП. Частота дискретизации для импульсной компоненты составляет $12.5~\text{нс}$, для постоянной --- показания АЦП сохраняются поминутно.

\subsubsection{Линейность}

Линейность системы регистрации света (связки ФЭУ и считывающей аппаратуры) была отдельно исследована по калибровочным кадрам. Интенсивность света в них намного выше даже самого яркого события ШАЛ, но даже в этих условиях амплитудная характеристика остаётся весьма близка к линейной, и может быть подвергнута дополнительной небольшой коррекции \cite[рис. 6]{SphereCalibration2016}. Поэтому в дальнейшем система регистрации света считается линейной.
